\section{Country-Specific DNS Censorship Trends and Mechanisms}
\label{sec:cases}



\paragraph{Iran}
While Iran supports both IPv4 and IPv6, it is more effective at blocking IPv4
resolvers (in both \texttt{A} and \texttt{AAAA} records).
Notably 272 of the 277 IPv6 resolvers in Iran rely on 6to4 bridges, all of which
block IPv6 queries at slightly lower rates than IPv4 queries sent to their pair address.
However, IPv6 queries are censored by resolvers using 6to4 at a slightly higher
rate than resolvers using native IPv6 -- though the
sample size for native IPv6 is small (5). One potential explanation is that
public 6to4 appliances in Iran may be infrastructural, and hosted by the state-run ISPs.
The 6to4 protocol provides a bridge to allow disparate IPv6 networks
to communicate over the well connected IPv4 Internet using packet encapsulation.
For example, sending a DNS request over IPv6 to \texttt{2002:0102:0304::} will
send an \texttt{[IPv6|UDP|DNS]} packet over IPv6 to the 6to4 gateway at that address
which will then encapsulate the packet in IPv4 to 1.2.3.4 (
    whose 32-bit address is encoded in the IPv6 address \texttt{0102:0304}) in order to transit a portion of the
network with no connectivity. For this portion of the journey the packet is
\texttt{[IPv4|IPv6|UDP|DNS]} until it reaches the destination or another
network capable of routing IPv6 packets.

Further manual inspection shows that although most of Iran's resolvers belong
to the non-corporate type, the censorship rate remains similar across
corporate and non-corporate type resolvers (around ~23\%) which points toward a
centralized censorship policy within the country.


\paragraph{Thailand} One prominent example is Thailand, where 8.2\% of
\texttt{A} records were blocked by IPv4 resolvers, compared to around 1\% for
\texttt{AAAA} records or IPv6 resolvers in the country. The reason for this is
that many instances of IPv4 resolvers predominantly (or exclusively) saw
blocking on only \texttt{A} records, while their IPv6 pair saw little or no
blocking on either \texttt{A} or \texttt{AAAA} records. For instance, one of the
larger resolvers by blocked domains had an IPv4 endpoint that blocked
240 \texttt{A} records, but only 15 \texttt{AAAA} records from our domain list.
Meanwhile, its IPv6 counterpart did not block any domains (either \texttt{A} or
\texttt{AAAA} records). Furthermore, number of requests censored by connection type
closely matches the proportion of the resolvers of that connection type. For example,
~74\% of all resolvers in Thailand are of type Cable/DSL and they are responsible for
~72\% of all blocked queries. This coupled with findings in 
\Cref{sec:resources:resolvers} 
 (Thailand having a low $\nabla_{net}$ presents an even stronger case for a centralized 
 mechanism for censorship. We also found that out of the 184 total resolver pairs
 in Thailand, IPv6 resolvers of around 112 pairs are 6to4 bridges which again show a 
 very low censorship rate of around 0.64\% for \texttt{AAAA} records as opposed to the 
 censorship rate of 7.84\% of their IPv4 resolver counterparts. The censorship rate
 for native IPv6 resolvers was also low (1.34\%) but almost double that of its 6to4
 bridge counterparts.


% jq -r 'select(.resolver_country=="TH" and .id=="4934-A").blocked_domains | keys | .[]' ../../data/resolver-blocks-jan-aaaa.json | grep -e '-AAAA$' | wc -l
% Removed 3 domains manually: www.stratcom.mil, www.hud.gov, portal.hud.gov




\paragraph{China}
China shows an unusual preference \textbf{toward} blocking \texttt{AAAA}
records over \texttt{A} records. While it is known that the Great Firewall
can block \texttt{AAAA} records and injects IPv6 traffic, it is not clear why it
would block those records more than \texttt{A} records. Manual investigation reveals
21 domains that almost exclusively have their \texttt{AAAA} record blocked, but
not their \texttt{A} record. For example, \texttt{gmail.com}'s \texttt{AAAA}
record is blocked by over 95\% of resolvers in China, but the corresponding
\texttt{A} record for \texttt{gmail.com} is only blocked by 1\% of resolvers.
The other 20 domains have similar patterns, leading to China's slight preference
in blocking \texttt{AAAA} records over \texttt{A} records. We do not find any
instances of domains in China that are similarly exclusively blocked by
\texttt{A} record but not \texttt{AAAA}. As established in 
\Cref{sec:resources:resolvers},
China has a high ($\nabla_{net}$) which suggests that there are individual resolver
level inconsistencies. 90\% of all resolvers in China fall within one standard deviation
of the mean number of censored queries by all resolvers. Resolvers that censor
more than one standard deviation of the mean of requests consist disproportionately
of resolvers belonging to the cellular connection type. Although cellular type resolvers
constitute of 17\% of all resolvers in China, they represent 50\% of the resolvers which
censor more than one standard deviation above the mean number of censored queries. 
When compared to China and Thailand, China has a much larger percentage of native
IPv6 resolvers but it is still the most consistent in its censorship rate for \texttt{AAAA}
queries over both 6to4 and native IPv6 resolvers (~30\%) when compared with their IPv4
counterparts. However, the censorship rate for 6to4 when compared with native IPv6 resolvers
was still slightly lower (28\% vs 32\%).
%{"domain":"www.gmail.com-AAAA","country_code":"CN","v4_censored_count":186,"v6_censored_count":185}
%{"domain":"www.gmail.com-A","country_code":"CN","v4_censored_count":4,"v6_censored_count":1}

%{"domain":"www.apple.com-AAAA","country_code":"CN","v4_censored_count":189,"v6_censored_count":191}
%{"domain":"www.apple.com-A","country_code":"CN","v4_censored_count":5,"v6_censored_count":8}

% ~/v6/v4vsv6/cmd/baseRate$ python3 q6.py < ./CN.json 
%blocked in AAAA but not A: 21 domains
%translate.google.com
%learndigital.withgoogle.com
%www.snapchat.com
%news.google.com
%play.google.com
%video.google.com
%picasa.google.com
%www.projectbaseline.com
%googleblog.com
%hangouts.google.com
%messages.android.com
%www.apple.com
%android.com
%google.dz
%raseef22
%qalam.withgoogle.com
%www.gmail.com
%doubleclick.net
%gmail.com
%www.hacktivismo.com
%allo.google.com
%blocked in A but not AAAA: 0 domains



%The reason
%that these hosts still see blocking at all is that they occasionally must
%recursively resolve the host, which may induce censorship on the outbound
%packet. % (how did they get a correct cached entry in the first place tho?)

% $ cat ../../data/resolver-blocks-jan-aaaa.json | jq '[.resovler_ip, .resolver_country] | @tsv' -r | awk '{print $1, $2}' | grep '2002:' | awk '{print $2}' | sort | uniq -c | sort -rn | head
%    628 KR
%    467 US
%    364 FR
%    272 IR
%    255 RU
%    244 DE
%    225 VN
%    175 IN
%    146 CA
%    127 GB
%



% 5.56.132.126 2002:538:847e::538:847e

\paragraph{Trends}
We observe two general trends that apply to many censoring countries in our data.
First, \textbf{IPv4 resolvers are more heavily censored than IPv6 resolvers}.
This may be because censors in those regions only inspect IPv4 packets, or that
censors are more widely deployed on IPv4 networks. Second, \textbf{Native record
types are more heavily censored than non-native records}. In other words, an
\texttt{A} record requested from an IPv4 resolver is more likely to be censored
than a \texttt{AAAA} record from the same interface. But conversely, we find
that \texttt{AAAA} records are also more often censored on IPv6 resolvers than
\texttt{A} records. This could be because a censor only supports detecting
\texttt{AAAA} records in IPv6 traffic, and only \texttt{A} records in IPv4.


\if0
Notably, China, Russia, Iran, and Hong Kong can clearly be seen to host
resolvers that consistently block many domains, demonstrating these countries have
deployed largely uniform national censorship policies. On the other hand,
countries that are known to censor may still appear to have no uniformly blocked
domains. This could be due to a country primarily using a different technology
to censor (such as HTTP, SNI, or IP blocking), bias in the set of domains we
queried, or in the set of resolvers used.

One illustrative example is India, a country where widespread Internet censorship has
been observed and studied~\cite{singh2020india,Yadav2018a}. Given the number
of resolvers we discover and the country's proclivity to censor, we would expect to see
some domains censored across the country, but instead our data suggests no
domains are censored uniformly. One explanation for this discrepancy is that
India's censorship is not as centralized as in other countries: private ISPs in
India are given lists of URLs to block, but it is up to the ISP how to carry out
this blocking~\cite{Gosain2017a}, leading to heterogeneity in what is blocked,
what techniques are used, and how often updates occur.  Figure~\ref{fig:india}
shows a clustering of Indian resolvers by what domains they block, illustrating the non-uniformity in
conducting censorship in the country. We draw resolvers as nodes sized proportional
to their blocklist size, and draw an edge between two resolver
nodes if the set of domains they block has a high similarity (Levenshtein edit
distance less than a threshold). The largest censored list found in India contained 31
censored domains, while many other resolvers censored few or no domains.
% jq 'select(.resolver_country=="IN" and .id=="4335-A").blocked_domains | keys | .[]' ../../data/resolver-blocks-jan-aaaa.json -r | awk -F'-' '{print $1}' | sort | uniq | wc -l


\FigIndiaCluster
\fi



\section{DNS Censorship Trends and Mechanisms}
\label{sec:cases}
In this section, we provide interesting findings about the censorship
infrastructure in three heavily censored regions which have high IPv6 adoption
and have not been discussed in detail in prior sections: Iran, Thailand, and
China.

% \paragraph{6to4 protocol}
% Our country specific dives reference 6to4 bridges which we first explain here.
% The 6to4 protocol provides a bridge to allow disparate IPv6 networks to
% communicate over the well connected IPv4 Internet using packet encapsulation.
% For example, sending a DNS request over IPv6 to \texttt{2002:0102:0304::} will
% send an \texttt{[IPv6|UDP|DNS]} packet over IPv6 to the 6to4 gateway at that
% address which will then encapsulate the packet in IPv4 to 1.2.3.4 (whose 32-bit
% address is encoded in the IPv6 address \texttt{0102:0304}) in order to transit a
% portion of the network with no connectivity. For this portion of the journey the
% packet is \texttt{[IPv4|IPv6|UDP|DNS]} until it reaches the destination or
% another network capable of routing IPv6 packets.
% 

\para{Iran.}
While Iran supports both IPv4 and IPv6, it is more effective at blocking queries
from IPv4 resolvers (in both \texttt{A} and \texttt{AAAA} records). Notably, 272
of the 277 IPv6 resolvers in Iran rely on 6to4 bridges, all of which block
queries over IPv6 at statistically significantly lower rates (-3.2\%) than
queries over IPv4. 
%
However, we find that queries over IPv6 are censored by resolvers using 6to4
bridges at nearly the same rate as resolvers that use native IPv6 connections
(note that our sample size of 5 native IPv6 resolvers in Iran is too small for
statistical validation). This is a rather surprising result since one expects
that the censorship rate of IPv4 should be independent of whether the IPv6 is
native or 6to4 bridged. One potential explanation is that public 6to4
appliances in Iran may be infrastructural, and hosted by the state-run ISPs.
See \Cref{tab:prevalence:rates:broken} (Appendix) for a breakdown of
censorship rates by countries with 6to4 bridges.
%
We also perform an inspection of the rates of censorship in corporate and
non-corporate networks in Iran. We find that there are no significant
differences between them (~23\%). This points towards a centralized censorship
apparatus within the country.

\para{Thailand.} 
In Thailand, we see significant differences in the handling of {\tt A} queries
sent over IPv4 and all other query-network combinations --- 8.2\% of \texttt{A}
queries sent to IPv4 resolvers were censored, compared to $\approx$ 1\% of all
other query and network combinations. 
% The reason for this is that many instances of IPv4 resolvers
% predominantly (or exclusively) saw blocking on only \texttt{A} records, while
% their IPv6 pair saw little or no blocking on either \texttt{A} or \texttt{AAAA}
% records. 
For instance, examining the IPv4 interface of one of the resolvers with the
largest blocked domains list: it blocked 240 \texttt{A} records, but only 15
\texttt{AAAA} records from our domain list. Meanwhile, its IPv6 counterpart did
not block any domains (either \texttt{A} or \texttt{AAAA} records). These
findings suggest an incomplete DNS censorship infrastructure that is yet to
effectively handle IPv6 traffic.
%
Furthermore, the number of requests censored broken down by connection type
closely matches the proportion of the resolvers of that connection type. For
example, ~74\% of all resolvers in Thailand are of type Cable/DSL and they are
responsible for ~72\% of all blocked queries. This, coupled with findings in
\Cref{sec:infrastructure:resolvers} (Thailand having a low $\nabla_{net}$)
present a strong case for a centralized mechanism for censorship. 
%
Despite the already low rate of native IPv6 censorship (1.3\%), we find that
the 112 6to4-bridged resolvers had an even lower rate of censorship (0.6\%)
which suggests, once again, an inadequately equipped IPv6 censorship mechanism.


% jq -r 'select(.resolver_country=="TH" and .id=="4934-A").blocked_domains | keys | .[]' ../../data/resolver-blocks-jan-aaaa.json | grep -e '-AAAA$' | wc -l
% Removed 3 domains manually: www.stratcom.mil, www.hud.gov, portal.hud.gov




\para{China.}
China is unique in that it shows an unusual preference \textbf{toward} blocking
\texttt{AAAA} records over \texttt{A} records. While it is known that the Great
Firewall can block \texttt{AAAA} records and injects IPv6 traffic, it is not
clear why it would block those records more than \texttt{A} records. Manual
investigation reveals 21 domains that almost exclusively have their
\texttt{AAAA} record blocked, but not their \texttt{A} record. For example,
\texttt{gmail.com}'s \texttt{AAAA} record is blocked by over 95\% of resolvers
in China, but the corresponding \texttt{A} record for \texttt{gmail.com} is
only blocked by 1\% of resolvers. The other 20 domains have similar patterns,
leading to China's slight preference in blocking \texttt{AAAA} records over
\texttt{A} records. We do not find any instances of domains in China that are
similarly exclusively blocked by \texttt{A} record but not \texttt{AAAA}. 
%
As discussed in \Cref{sec:infrastructure:domains}, China has a high
$\nabla_{net}$ which suggests that there are individual resolver-level
inconsistencies. 90\% of all resolvers in China fall within one standard
deviation of the mean number of censored queries by all resolvers in China.
Those that exceed these bounds are overwhelmingly classified as `Cellular' by
Maxmind's connection type database. Although they make up just 17\% of all our
resolvers in China, they present 50\% of the resolvers with censorship rates
higher than mean+std.
%
China has a much larger percentage of native IPv6 resolvers compared to our
other case studies and it is still the most consistent in its censorship rate
for \texttt{AAAA} queries over both 6to4 and native IPv6 resolvers (~30\%) when
compared with their IPv4 counterparts. However, the censorship rate for 6to4
when compared with native IPv6 resolvers was still slightly lower (28\% vs
32\%). 
%
Taken all together, China appears to be the best equipped IPv6 censor with gaps
only arising in the IPv4 censorship apparatus.

%{"domain":"www.gmail.com-AAAA","country_code":"CN","v4_censored_count":186,"v6_censored_count":185}
%{"domain":"www.gmail.com-A","country_code":"CN","v4_censored_count":4,"v6_censored_count":1}

%{"domain":"www.apple.com-AAAA","country_code":"CN","v4_censored_count":189,"v6_censored_count":191}
%{"domain":"www.apple.com-A","country_code":"CN","v4_censored_count":5,"v6_censored_count":8}

% ~/v6/v4vsv6/cmd/baseRate$ python3 q6.py < ./CN.json 
%blocked in AAAA but not A: 21 domains
%translate.google.com
%learndigital.withgoogle.com
%www.snapchat.com
%news.google.com
%play.google.com
%video.google.com
%picasa.google.com
%www.projectbaseline.com
%googleblog.com
%hangouts.google.com
%messages.android.com
%www.apple.com
%android.com
%google.dz
%raseef22
%qalam.withgoogle.com
%www.gmail.com
%doubleclick.net
%gmail.com
%www.hacktivismo.com
%allo.google.com
%blocked in A but not AAAA: 0 domains



%The reason
%that these hosts still see blocking at all is that they occasionally must
%recursively resolve the host, which may induce censorship on the outbound
%packet. % (how did they get a correct cached entry in the first place tho?)

% $ cat ../../data/resolver-blocks-jan-aaaa.json | jq '[.resovler_ip, .resolver_country] | @tsv' -r | awk '{print $1, $2}' | grep '2002:' | awk '{print $2}' | sort | uniq -c | sort -rn | head
%    628 KR
%    467 US
%    364 FR
%    272 IR
%    255 RU
%    244 DE
%    225 VN
%    175 IN
%    146 CA
%    127 GB
%



% 5.56.132.126 2002:538:847e::538:847e

% \paragraph{Trends}
% We observe two general trends that apply to many censoring countries in our data.
% First, \textbf{IPv4 resolvers are more heavily censored than IPv6 resolvers}.
% This may be because censors in those regions only inspect IPv4 packets, or that
% censors are more widely deployed on IPv4 networks. Second, \textbf{Native record
% types are more heavily censored than non-native records}. In other words, an
% \texttt{A} record requested from an IPv4 resolver is more likely to be censored
% than a \texttt{AAAA} record from the same interface. But conversely, we find
% that \texttt{AAAA} records are also more often censored on IPv6 resolvers than
% \texttt{A} records. This could be because a censor only supports detecting
% \texttt{AAAA} records in IPv6 traffic, and only \texttt{A} records in IPv4.


\if0
Notably, China, Russia, Iran, and Hong Kong can clearly be seen to host
resolvers that consistently block many domains, demonstrating these countries have
deployed largely uniform national censorship policies. On the other hand,
countries that are known to censor may still appear to have no uniformly blocked
domains. This could be due to a country primarily using a different technology
to censor (such as HTTP, SNI, or IP blocking), bias in the set of domains we
queried, or in the set of resolvers used.

One illustrative example is India, a country where widespread Internet censorship has
been observed and studied~\cite{singh2020india,Yadav2018a}. Given the number
of resolvers we discover and the country's proclivity to censor, we would expect to see
some domains censored across the country, but instead our data suggests no
domains are censored uniformly. One explanation for this discrepancy is that
India's censorship is not as centralized as in other countries: private ISPs in
India are given lists of URLs to block, but it is up to the ISP how to carry out
this blocking~\cite{Gosain2017a}, leading to heterogeneity in what is blocked,
what techniques are used, and how often updates occur.  Figure~\ref{fig:india}
shows a clustering of Indian resolvers by what domains they block, illustrating the non-uniformity in
conducting censorship in the country. We draw resolvers as nodes sized proportional
to their blocklist size, and draw an edge between two resolver
nodes if the set of domains they block has a high similarity (Levenshtein edit
distance less than a threshold). The largest censored list found in India contained 31
censored domains, while many other resolvers censored few or no domains.
% jq 'select(.resolver_country=="IN" and .id=="4335-A").blocked_domains | keys | .[]' ../../data/resolver-blocks-jan-aaaa.json -r | awk -F'-' '{print $1}' | sort | uniq | wc -l


\FigIndiaCluster
\fi



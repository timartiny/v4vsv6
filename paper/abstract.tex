

Internet censorship impacts large segments of the Internet, but so far, prior
work has focused almost exclusively on performing measurements using IPv4. As the
Internet grows, and more users connect, IPv6 is increasingly supported and
available to users and servers alike. But despite this steady growth, it remains
unclear if the information control systems that implement censorship (firewalls,
deep packet inspection, DNS injection, etc) are as effective with IPv6 traffic
as they are with IPv4.

In this paper, we perform the first global measurement of DNS censorship on the
IPv6 Internet. Leveraging a recent technique that allows us to discover
dual-stack IPv6 open resolvers (along with their corresponding IPv4 address), we send
over 20~million {\tt A} and {\tt AAAA} DNS requests to DNS resolvers worldwide,
and measure the rate at which they block, at the resolver, network, and country
level.

We observe that while nearly all censors support blocking IPv6, their policies
are inconsistent with and frequently less effective than their IPv4 censorship
infrastructure. Our results suggest that supporting IPv6 censorship is not
all-or-nothing: many censors support it, but poorly.  As a result, these censors
may have to expend additional resources to bring IPv6 censorship up to parity
with IPv4. In the meantime, this affords censorship circumvention researchers a
new opportunity to exploit these differences to evade detection and blocking.


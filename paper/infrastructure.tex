\section{Censorship of IPv4 and IPv6 DNS Traffic}
\label{sec:infrastructure}

\para{Overview.} In this section, we focus on {\it identifying and
characterizing the differences in handling DNS queries sent over IPv4 and IPv6
networks} in DNS censorship mechanisms. 
%
Specifically, we answer the following questions:
%
(\Cref{sec:infrastructure:country}) In which countries are the DNS censorship
mechanisms for IPv4 and IPv6 traffic significantly different?,
%
(\Cref{sec:infrastructure:resolvers}) what are the characteristics of resolvers
that exhibit differences in the censorship of IPv4 and IPv6 traffic?, and
%
(\Cref{sec:infrastructure:domains}) what are the characteristics of the domains
in which such differences are frequently exhibited?

\subsection{Within-country differences in the censorship of IPv4 and IPv6 DNS
queries} \label{sec:infrastructure:country}

We use the responses received from the IPv4 and IPv6 interfaces of each of the
resolvers in our dataset for the same set of domains. We then apply
the censorship detection mechanism detailed in \Cref{sec:methodology:censorship} to
measure the prevalence of DNS censorship of queries sent over IPv4 and IPv6.
Finally, we perform statistical tests to identify the countries that have
significant differences in their censorship of IPv4 and IPv6 DNS traffic.

\para{Identifying differences within a country.} 
To measure differences in the censorship of DNS queries sent over IPv4 and
IPv6, we compare the prevalence of censorship on each by aggregating their
responses across each resolver within a country. This presents us with two
distributions (corresponding to the IPv4 and IPv6 interfaces of resolvers) of
the fraction of censored domains from resolvers within the corresponding
country.
%
We use a two-sample $t$-test to verify statistical significance of any observed
differences between the two groups for each country. Similar to our approach in
\Cref{sec:resources}, we apply a Sidak correction to control for Type I errors
from multiple hypothesis testing. 
%
This requires $p \leq 1-{.05}^{1/n_{c}}$ for classifying a difference as
significant, where $n_c$ is the total number of countries in our dataset
(106). 
%
The presence of a statistically significant difference for a specific country
would imply that the country appears to have different censorship mechanisms
for IPv4 and IPv6 DNS traffic (if a centralized mechanism for censorship
exists) or that a significant number of resolvers within that country are not
consistent in their censorship of IPv6 and IPv4 traffic.
%
A summary of our results are presented in \Cref{tab:infrastructure:countries}.

\para{Which countries demonstrate large-scale inconsistencies in their
handling of IPv4 and IPv6 DNS traffic?}
%
In total, we find only five countries (Thailand, Iran, Bangladesh, Myanmar, and
the United States) with statistically significant differences in their handling
of DNS queries over IPv4 and IPv6 traffic --- suggesting the use of independent
censorship mechanisms for IPv4 and IPv6. 
%
Interestingly, all these countries appear to have gaps in their IPv6 censorship
apparatus --- \ie IPv4 rates of blocking are higher than IPv6 rates in all
countries with significant differences. These differences result in IPv6
queries experiencing between 12\% and 78\% less censorship than their IPv4
counterparts.
%
Once again, this suggests a tendency for network operators to more effectively
maintain IPv4 DNS censorship infrastructure than IPv6 infrastructure. These
gaps present opportunities for the success of circumvention tools with IPv6
capabilities.  
%
Further analysis shows that these differences primarily arise due to the fact
that DNS type {\tt A} queries are significantly more likely to be blocked over
IPv4 connections than over an IPv6 connection. However, this is not the case
for {\tt AAAA} queries, with the exception of Iran.
%
Taken together, these findings are particularly noteworthy for circumvention
efforts in Thailand, Myanmar, and Iran where IPv6 adoption rates are high
(between 15\% and 45\%) and dual-stack tools may be used for circumvention of
DNS censorship.
%

\begin{table}[t]
  \centering
  \small
  \scalebox{\tabularscale} {
  \begin{tabular}{lccc}%p{.9in}p{.9in}}
    \toprule
    {\bf Country}&{\bf {\tt A} queries }&{\bf {\tt AAAA} queries} & {\bf All queries}
    \\ \midrule
    Thailand (TH)      & -7.1 pp (-86.3\%) & $ns$              & -3.7 pp (-78.1\%) \\
    Iran (IR)          & -3.2 pp (-12.6\%) & -3.0 pp (-12.5\%) & -3.1 pp (-12.5\%) \\ 
    Bangladesh (BD)    & -5.5 pp (-86.1\%) & $ns$              & -3.0 pp (-77.9\%) \\
    Myanmar (MY)       & -3.6 pp (-74.2\%) & $ns$              & -2.1 pp (-62.3\%) \\
    United States (US) & -0.9 pp (-64.8\%) & $ns$              & -0.5 pp (-42.6\%) \\
    \midrule
    Korea (KR)         & -1.1pp (-46.5\%) & $ns$    & $ns$ \\
    Chile (CL)         & -1.6pp (-58.7\%)  & $ns$    & $ns$ \\
    \bottomrule
  \end{tabular}
  }
  \caption{Differences in blocking rates of DNS queries sent to IPv4 and IPv6
  interfaces of each resolver in a country. `pp' denotes the change in
  terms of percentage points (computed as blocking rate of IPv6 - blocking
  rate of IPv4) and the \%age value denotes the percentage change in blocking rate
  (computed as 
  $
  100 \times \frac{\text{IPv6 blocking rate} - \text{IPv4 blocking rate}}
  {\text{IPv4 blocking rate}}
  $). 
  Only countries having a statistically
  significant difference are reported. A negative value indicates that queries
  sent over IPv4 observed higher blocking rates than those sent over IPv6. $ns$
  indicates the difference was not statistically significant and thus omitted.}
  \label{tab:infrastructure:countries}
\end{table}

\subsection{Characteristics of IPv4/IPv6-inconsistent resolvers}
\label{sec:infrastructure:resolvers}
%
Our previous results demonstrate the promise of using IPv6 channels for DNS
resolution to bypass IPv4 DNS censorship. We now focus on identifying the
distribution and connection-type of resolvers demonstrating inconsistencies in
their handling of IPv4 and IPv6 traffic. This serves two purposes.
%
First, our analysis on the distribution of IPv4/IPv6-inconsistent resolvers
provides evidence-driven hypotheses about how information controls deployments
are structured in different countries. 
%
Second, studying the connection-types of IPv4/IPv6-inconsistent resolvers sheds
light on whether the gaps in censorship are visible to users in residential
networks. This serves as an indicator for the potential gains to be had by
circumvention tools that begin exploiting the IPv4/IPv6 gap.

\para{Identifying IPv4/IPv6-inconsistent resolvers.} Our approach is similar to
the methods used to identify {\tt A/AAAA}-inconsistent resolvers
(\cf \Cref{sec:resources:resolvers}). 
%
We compare the ratios of censored responses received from a single resolver
pair's IPv4 and IPv6 interfaces. We test whether these ratios are statistically
different using a $z$-test with a Sidak corrected $p \leq 1-.05^{1/n_{r_c}}$
being required for a statistically significant difference. 
%
A summary of the characteristics of the inconsistent resolvers identified in
each country is illustrated in \Cref{tab:infrastructure:resolvers}.

\para{Which countries have the largest fraction of IPv4/IPv6-inconsistent
resolvers?}
%
Two countries from our previous analysis on {\tt A/AAAA}-inconsistencies once
again appear with a large fraction of IPv4/IPv6-inconsistent resolvers ---
Thailand (81\%) and Bangladesh (65\%). Myanmar presents a new addition with
60\% of its resolvers demonstrating IPv4/IPv6-inconsistencies. Other countries
were found to have smaller fractions ranging from 12-26\%.

\para{How spread out are the IPv4/IPv6-inconsistent resolvers?}
%
In order to characterize the spread of IPv4/IPv6-inconsistent resolvers within
a country, we compute the entropy of the AS distribution of all resolvers and
IPv4/IPv6-inconsistent resolvers within a country
($S^{\text{all}}_{\text{net}}$ and $S^{\text{inconsistent}}_{\text{net}}$) and
then compute the KL-divergence of the distribution of inconsistent resolvers
from the distribution of all resolvers in that country ($\nabla_{net}$).
%
Similar to before, a large change in $\nabla_{net}$ means that the
IPv4/IPv6-inconsistencies arise from a small fraction of ASes and would suggest
that the gaps exist due to local network/resolver misconfigurations --- as
would be the case if regional operators implement their own DNS censorship
mechanisms. Conversely, a small change means that the gaps that exist roughly
equally impact all the ASes having resolvers and would suggest that the gaps
exist due to misconfigurations in a centralized DNS censorship mechanism.
%
Our results once again suggest the presence of a centralized blocking mechanism
in Thailand, Bangladesh, and Myanmar ($\nabla_{net} \in [0.13, 0.48]$) which
causes the IPv4/IPv6-inconsistencies. The United States has the highest
$\nabla_{net}$ observed which indicates that regional policies are responsible
for the IPv4/IPv6-inconsistencies. 

\para{What types of networks exhibit the most IPv4 and IPv6 inconsistencies?}
%
An overwhelming majority of the inconsistent resolvers in Thailand, Iran, and
Bangladesh (77\%-100\%) are found to be present in networks with (Maxmind
categorized) Cable/DSL connection-types that are typically associated with
residential networks. 
%
Put in the context of our previous result which suggests the presence of
a centralized DNS censorship mechanism in Thailand and Bangladesh, this
suggests that the IPv4/IPv6 gaps that exist in this mechanism also extend to
residential networks in the country.
%
Myanmar and the United States experience such inconsistencies primarily due to
their corporate networks which contain between 67-87\% of their inconsistent
resolvers. 

\begin{table*}[t]
  \centering
  \small
  \scalebox{\tabularscale} {
    \begin{tabular}{lcclcccl}
    \toprule
      {\bf Country} & {\bf Total pairs} & {\bf Inconsistent pairs} & {\bf Most inconsistent AS} & \multicolumn{3}{c}{\bf AS diversity} & {\bf Most inconsistent type} \\
      & & {(\% of total pairs)}& {(\# inconsistent pairs)}
      & $S^{\text{all}}_{\text{net}}$ & $S^{\text{inconsistent}}_{\text{net}}$
      & $\nabla_{\text{net}}$  & {(\# inconsistent pairs)} \\
      \midrule
      Thailand (TH)            & 186    & 151 (81.2\%)  & AS 9835 Government IT Services (39)  & 4.50 & 4.10 & 0.13 & Cable/DSL (108) \\
      Iran (IR)                & 277    & 74 (26.7\%)   & AS 208161 PARSVDS (11)               & 5.03 & 3.81 & 0.87 & Cable/DSL (57) \\
      Bangladesh (BD)          & 29     & 19 (65.2\%)   & AS 9230 Bangladesh Online (4)        & 4.10 & 3.72 & 0.39 & Cable/DSL (19) \\
      Myanmar (MY)             & 50     & 30 (60.0\%)   & AS 136170 Exabytes Network (10)      & 3.31 & 2.42 & 0.48 & Corporate (26) \\
      United States (US)       & 1,228  & 151 (12.3\%)  & AS 30457 WEHOSTWEBSITES (36)         & 6.28 & 5.22 & 1.44 & Corporate (102) \\
      \midrule                 
      Korea (KR)               & 632    & 101 (16.0\%)  & AS 9848 Sejong Telecom (13)          & 3.12 & 4.14 & 0.95 & Cable/DSL (73)   \\
      Chile (CL)               & 65     & 16 (24.6\%)   & AS 27651 Entel Chile (12)            & 3.08 & 1.19 & 1.04 & Corporate (11) \\
    \bottomrule
  \end{tabular}
  }
  \caption{Characteristics of the resolvers which demonstrated a statistically
  significant difference in their handling of DNS queries over IPv4 and IPv6
  each country.
  %
  `AS diversity' denotes the entropies of (all) resolver distribution
  ($S^{\text{all}}_{\text{net}}$) and {IPv4/IPv6}-inconsistent resolver
  distribution ($S^{\text{inconsistent}}_{\text{net}}$) across a country's
  ASes, and `$\nabla_{\text{net}}$' represents the Kullback-Leibler divergence
  of the distribution of inconsistent resolvers from the distribution of all
  resolvers in the country's ASes (\cf \Cref{sec:infrastructure:resolvers}).
  %
  `Most inconsistent type' denotes the connection type with the most number of
  {IPv4/IPv6}-inconsistent resolvers.}
  \label{tab:infrastructure:resolvers}
\end{table*}


\subsection{Characterization of anomalous domains}
\label{sec:infrastructure:domains}


\section{Censorship of IPv4 and IPv6 DNS Traffic}
\label{sec:infrastructure}

\para{Overview.} In this section, we focus on {\it identifying and
characterizing the differences in handling DNS queries sent over IPv4 and IPv6
networks} in DNS censorship mechanisms. 
%
Specifically, we answer the following questions:
%
(\Cref{sec:infrastructure:country}) In which countries are the DNS censorship
mechanisms for IPv4 and IPv6 traffic significantly different?,
%
(\Cref{sec:infrastructure:resolvers}) what are the characteristics of resolvers
that exhibit differences in the censorship of IPv4 and IPv6 traffic?, and
%
(\Cref{sec:infrastructure:domains}) what are the characteristics of the domains
in which such differences are frequently exhibited?

\subsection{Within-country differences in the censorship of IPv4 and IPv6 DNS
queries} \label{sec:infrastructure:country}

We use the responses received from the IPv4 and IPv6 interfaces of each of the
dual-stack resolvers in our dataset for the same set of domains. We then apply
the censorship detection mechanism detailed in \Cref{sec:methods:labeling} to
measure the prevalence of DNS censorship of queries sent over IPv4 and IPv6.
Finally, we perform statistical tests to identify the countries that have
significant differences in their censorship of IPv4 and IPv6 DNS traffic.

\para{Identifying differences with a country.} 

\para{Results.}

\section{Censorship of IPv4 and IPv6 Resource Records} \label{sec:resources}

\para{Overview.}
In this section, we focus on {\it identifying and characterizing differences in
the handling of IPv4 and IPv6 resource records} in DNS censorship deployments.
Specifically, we seek to answer the following questions: 
%
(\Cref{sec:resources:country}) In which countries is the censorship of IPv4
\textbf{resource records} (DNS {\tt A} queries) significantly different than the
censorship of IPv6 resource records (DNS {\tt AAAA} queries)?,
%
(\Cref{sec:resources:resolvers}) what are the characteristics of the
\textbf{resolvers} which exhibit differences in the handling of {\tt A} and {\tt
AAAA} queries?, and 
%
(\Cref{sec:resources:domains}) what are the characteristics of \textbf{domains}
in which these differences are frequently observed?
%

%\subsection{Within-country differences in the censorship of
%\texttt{A} and \texttt{AAAA} resource records} \label{sec:resources:country}
\subsection{{\tt A} vs. {\tt AAAA} resource censorship} \label{sec:resources:country}
%
We use the responses received from our {\tt A} and {\tt AAAA} queries sent to
the same set of resolvers and for the same set of domains (\cf
\Cref{sec:methodology} for data collection methodology). We then apply the
censorship determination methods described in \Cref{sec:methodology:censorship}
to measure the prevalence of censorship on our {\tt A} and {\tt AAAA} DNS
queries. Finally, we perform statistical tests to identify significant
differences in the prevalence of censorship of {\tt A} and {\tt AAAA} queries
within each country.

\para{Identifying differences within a country.}
To measure differences in DNS query handling within a specific country, we
compare the prevalence of censorship on {\tt A} and {\tt AAAA} queries by
aggregating responses across {each resolver within the country}. This presents
us with two distributions (one each for the group of {\tt A} and {\tt AAAA}
queries) of the fraction of censored domains observed at each resolver in the
country.
%
We use a two-sample $t$-test to verify statistical significance of any observed
differences between the two groups for each country. In our statistical
analysis, we aim to achieve a significance level of 5\% ($p \leq$  .05)
\emph{over all our findings}. Therefore, we apply a \Sidak correction
\cite{abdi2007bonferroni} to control for Type I (false-positive) errors from
multiple hypothesis testing. 
%
This requires $p \leq 1-{.05}^{1/n_{c}}$ for classifying a difference as
significant, where $n_c$ is the total number of countries in our dataset
(106). This approach reduces the likelihood of false-positive reports
of within-country differences.
%
The presence of a statistically significant difference for a specific country
would imply that \texttt{A} and \texttt{AAAA} resource types appear to undergo
different censorship mechanisms within that country (if a centralized mechanism
for censorship exists) or that a significant number of resolvers within that
country have inconsistencies in their censoring of each query type.
%
A summary of our results are presented in \Cref{tab:resources:countries}. 

\para{How many countries demonstrate large-scale inconsistencies in their
handling of {\tt A} and {\tt AAAA} queries?} 
%
In total, only seven countries showed a statistically significant difference
in the rate at which {\tt A} and {\tt AAAA} DNS requests were blocked. We note
that this is a conservative lower-bound due to the statistical test used, which
minimizes false positive errors at the expense of false negatives.
%
This finding suggests the presence of independent censorship mechanisms for
handling each query type in the seven countries (Thailand, Bangladesh,
Pakistan, Chile, Vietnam, Korea, and China).
%
Of these, six (China being the only exception) were found to have lower
blocking rates for {\tt AAAA} queries than {\tt A} queries. In fact, the {\tt
AAAA} censorship rates were between 36-78\% lower than the {\tt A} censorship
rate suggesting that their censorship mechanisms for {\tt AAAA} queries that
are associated with IPv6 connectivity are still lagging. 
%
Further analysis shows that the differences are mostly found on the IPv4
interfaces of our resolvers (\cf {\it IPv4 resolvers} column in
\Cref{tab:resources:countries}) where the {\tt AAAA} censorship rates were up to
86\% lower than the {\tt A} censorship rates. 
%
This finding is indicative of a tendency for network operators to have
focused efforts on maintaining infrastructure for censoring {\tt A} queries
sent to IPv4 resolvers, while paying less attention to the handling of {\tt
AAAA} queries and their IPv6 interfaces. It also presents an opportunity for
circumvention tool developers to exploit.
%
%Once again, 
China presents the only exception with a preference for blocking
{\tt AAAA} queries on both IPv4 and IPv6 interfaces of resolvers with a 10\%
and 13\% higher {\tt AAAA} censorship rate, respectively. We investigate this
anomaly in \Cref{sec:cases}.

\begin{table}[t]
  \centering
  \small
  \scalebox{\tabularscale} {
  \begin{tabular}{lccc}%p{.9in}p{.9in}}
    \toprule
    {\bf Country}&{\bf IPv4 resolvers}&{\bf IPv6 resolvers} & {\bf All resolvers}
    \\ \midrule
    Thailand (TH)      & -7.1 pp (-85.7\%) & $ns$               & -3.7 pp (-77.5\%) \\
    Bangladesh (BD)    & -5.1 pp (-80.0\%) & $ns$               & -2.6 pp (-71.3\%) \\
    Pakistan (PK)      & -2.1 pp (-73.6\%) & -2.8 pp (-59.8\%)  & -2.5 pp (-60.2\%) \\
    Chile (CL)         & -2.0 pp (-57.6\%) & $ns$               & -1.1 pp (-58.9\%) \\
    Vietnam (VN)       & $ns$              & -0.7 pp (-52.5\%)  & -0.7 pp (-47.1\%) \\
    Korea (KR)         & -1.3 pp (-54.6\%) & $ns$               & -0.6 pp (-36.2\%) \\
    China (CN)         &  3.1 pp (+10.4\%) &  3.7 pp (+12.9\%)  &  3.4 pp (+11.7\%) \\
    \midrule
    United States (US) & -0.5 pp (-33.2\%) &  0.4 pp (+72.3\%)  &  $ns$  \\
    Myanmar (MY)       & -2.9 pp (-58.9\%) & $ns$    &  $ns$  \\
    \bottomrule
  \end{tabular}
  }
  \caption{Differences in blocking rates of {\tt A} and {\tt AAAA} queries
  observed over IPv4, IPv6, and all resolvers in a country. `pp' denotes the
  change in terms of percentage points (computed as {\tt AAAA} blocking rate
  - {\tt A} blocking rate) and the \%age value denotes the percentage change in
  blocking rate (computed as
  $
  100\times\frac{\text{{\tt AAAA} blocking rate} - \text{{\tt A} blocking rate}}
  {\text{{\tt A} blocking rate}}
  $). 
  Only countries having a statistically significant difference are reported. A
  negative value indicates that {\tt A} queries observed higher blocking rates
  than {\tt AAAA} queries in a given country. $ns$ indicates the difference was
  not statistically significant and thus omitted.}
  \label{tab:resources:countries}
\end{table}

\subsection{Characteristics of {\tt A/AAAA}-inconsistent resolvers}
\label{sec:resources:resolvers}
Given our above results which suggest that there are a number of countries in
which {\tt A} and {\tt AAAA} queries are censored differently, we now seek to
understand the characteristics of the resolvers that cause these differences.
%
We first focus on identifying the individual resolvers in each country that
have statistically different behaviors for {\tt A} and {\tt AAAA} queries.
Then, we compare the AS distributions of these resolvers with the set of all
resolvers in a country. This comparison tells us if the {\tt A}/{\tt AAAA}
inconsistencies are specific to a subset of ISPs, or if the inconsistencies
exist across the whole country (suggesting centralized censorship).
Finally, we identify the
types of networks hosting inconsistent resolvers to get a measure of whether
users in residential networks may exploit these DNS inconsistencies for
circumvention.

\para{Identifying differences in individual resolvers.}
%
We begin our analysis by identifying the individual resolver pairs (\ie we
consider the IPv4- and IPv6-interfaces of a resolver as a unit), within each
of the seven countries listed above that have a statistically significant
difference in their censorship of {\tt A} and {\tt AAAA} queries.
%
To measure differences in DNS query handling of individual resolvers, we
compare the ratio of censored responses each resolver observes for {\tt A} and
{\tt AAAA} queries.

%observed from our {\tt A} and {\tt
%AAAA} queries {from each resolver}.
%
We use a two-proportion $z$-test to verify the statistical significance of any
observed difference in the ratios between the two groups for each resolver.
Similar to our within-country analysis, we apply a \Sidak correction to account
for our testing of multiple hypotheses and use $p \leq 1-{.05}^{1/n_{r_c}}$ to
classify a difference as significant, where $n_{r_c}$ is the total number of
resolvers in our dataset belonging to country $c$.
%
A summary of our results is provided in \Cref{tab:resources:resolvers}. 

\para{Which countries have the largest fractions of resolvers
exhibiting {\tt A} and {\tt AAAA} resolution inconsistencies?}
%
Immediately standing out from the other countries are Thailand, Bangladesh, and
Pakistan. These countries have {\tt A} and {\tt AAAA} inconsistencies in
62-82\% of their resolvers. In comparison, other countries with
statistically significant differences have inconsistencies arising from
anywhere between 3-30\% of their resolvers.
%

\para{How spread out are the {\tt A/AAAA}-inconsistent resolvers?} 
%
We calculate the entropy of the distribution of censorship by record query type
of all resolvers in the country ($S_{\text{query}}^{\text{all}}$) and compare it
with the entropy of the distribution of censorship by record query type of the
inconsistent resolvers in the country
($S_{\text{query}}^{\text{inconsistent}}$). This serves as a measure of the
diversity of ASes observed in both cases. 
%
In order to compare the two measures, we use the Kullback-Leibler divergence
($\nabla_{\text{query}}$) distance \cite{KLdivergence}. In simple terms, the
KL-divergence between two distributions ($X$, $Y$) measures the number
of additional bits required to encode $X$ given the optimal encoding for $Y$.
In other words, it is the relative entropy of one distribution given another. 
%If the distributions are similar, the number will be close to 0.
%
This computation is helpful for hypothesizing the censorship infrastructure
that causes the inconsistencies. Finding a small $\nabla_{\text{query}}$ value
in a country signifies that the inconsistent resolvers had a similar
distribution to all the resolvers in that country. This would suggest the
presence of a centralized mechanism that (roughly) equally impacts all ASes in
the country that is responsible for the inconsistencies.
Conversely, a higher $\nabla_{\text{query}}$ value indicates that there is
a strong change in the distribution of resolvers -- \ie a disproportionate
number of inconsistencies arise from a smaller set of ASes. This would be
indicative of local configuration inconsistency (at the network or resolver
level), rather than a centralized configuration inconsistency.
%
We note that this does not provide a measure of how centralized censorship is
within a country overall, but rather, it describes how uniform the {\tt A} vs.
{\tt AAAA} inconsistencies are.

%
Based on this analysis, we once again see that Thailand, Bangladesh, and
Pakistan stand out with small $\nabla_{\text{query}}$ values (0.14 - 0.48).
This suggests a country-wide censorship mechanism is responsible
for the inconsistent censorship of {\tt A} and {\tt AAAA} that impacts all
ASes nearly equally.
%
Korea, China, and the United States on the other hand demonstrate high
$\nabla_{\text{query}}$ scores suggesting the presence of network- or
resolver-level misconfigurations are responsible. This is confirmed by inspecting the ASes
hosting the resolvers with inconsistencies. For example, in the United States,
resolvers in just 5 ASes (of 249 ASes with resolvers) account for 56\% of all
{\tt A} and {\tt AAAA} inconsistencies.


\begin{table*}[t]
  \centering
  \small
  \scalebox{\tabularscale} {
    \begin{tabular}{lcclcccl}
    \toprule
      {\bf Country} & {\bf Total pairs} & {\bf Inconsistent pairs} & {\bf Most inconsistent AS} & \multicolumn{3}{c}{\bf AS diversity} & {\bf Most inconsistent type} \\ 
      & & {(\% of total pairs)}& {(\# inconsistent pairs)} & $S^{\text{all}}_{\text{query}}$ & $S^{\text{inconsistent}}_{\text{query}}$ & $\nabla_{\text{query}}$  & {(\# inconsistent pairs)} \\
      \midrule
      Thailand (TH)       & 186 & 152 (81.7\%)  &  AS9835 Government IT Services (40)  & 4.50 & 4.06 & 0.14 & Cable/DSL (110) \\
      Bangladesh (BD)     & 29  & 18 (62.1\%) & AS 9230 Bangladesh Online (4)          & 4.10 & 3.61 & 0.48 & Cable/DSL (18) \\    
      Pakistan (PK)       & 23  & 15 (65.2\%) & AS 17911 Brain Telecom (3)             & 3.43 & 3.06 & 0.25 & Cable/DSL (12) \\    
      Chile (CL)          & 65  & 20 (30.1\%) & AS 27651 Entel Chile (13)              & 3.08 & 1.14 & 1.18 & Corporate (13) \\    
      Vietnam (VN)        & 252 & 64 (25.4\%) & AS 131353 NhanHoa Software (37)        & 3.89 & 2.22 & 0.71 & Cable/DSL (59) \\    
      Korea (KR)          & 632 & 80 (12.7\%) & AS 9848 Sejong Telecom (13)            & 3.12 & 4.17 & 1.30 & Cable/DSL (58) \\    
      China (CN)          & 194 & 6 (3.1\%)   & AS 4538 China Education and Research Network (2) & 3.89 & 2.25 & 2.56 & Corporate (4)   \\    
    \midrule
      United States (US)  & 1,228 & 175 (14.3\%)  & AS 30475 WEHOSTWEBSITES (35) & 6.28 & 4.69 & 1.31 & Corporate (129) \\    
      Myanmar (MY)        & 50  & 30 (60\%)   & AS 136170 Exabytes Network (10)  & 3.31 & 2.42 & 0.48 & Corporate (26) \\    
    \bottomrule
  \end{tabular}
  }
  \caption{Characteristics of the resolvers which demonstrated a statistically
  significant difference in their handling of {\tt A} and {\tt AAAA} queries in
  each country. 
  %
  `AS diversity' denotes the entropies of (all) resolver distribution
  ($S^{\text{all}}_{\text{query}}$) and {\tt A/AAAA}-inconsistent resolver
  distribution ($S^{\text{inconsistent}}_{\text{query}}$) across a country's
  ASes, and `$\nabla_{\text{query}}$' represents the Kullback-Leibler
  divergence of the distribution of inconsistent resolvers from the
  distribution of all resolvers in the country's ASes (\cf
  \Cref{sec:resources:resolvers}).
  %
  `Most inconsistent type' denotes the connection type with the most number of
  {\tt A/AAAA}-inconsistent resolvers.}
  \label{tab:resources:resolvers}
\end{table*}

\para{What types of networks exhibit the most {\tt A} and {\tt AAAA}
inconsistencies?}
%
We use the Maxmind GeoIP2 connection type database (retrieved in 01/2022
\cite{maxmind-connectiondb}) to identify the connection type of the resolvers
responsible for {\tt A} and {\tt AAAA} inconsistencies. 
%
We find that, in most countries, Cable/DSL network connections (typically
associated with residential networks) were most likely to host a resolver
exhibiting an inconsistency. 
%
Of the seven countries with statistically significant overall differences, only
Thailand and China were found to have a high ratio of {\tt A/AAAA} -inconsistent
resolvers in corporate networks. 
%
Combined with our previous results which suggest the presence of an
inconsistency in a centralized mechanism in Thailand, Bangladesh, and Pakistan,
these results show that these inconsistencies are likely extending to
residential networks --- a promising sign for the citizen users of
circumvention tools which exploit the {\tt A/AAAA} gap.

\subsection{Characteristics of anomalous domains} 
\label{sec:resources:domains}

We now identify the category of domains that appear to exist in the gap of {\tt
A} and {\tt AAAA} blocking. In addition to providing a general categorization
of these domains, we also analyze whether their category distributions vary
significantly from the category distribution of websites that received any
blocking. We do this in order to identify the specific policies or mechanisms
that differ between the censorship mechanisms for {\tt A} and {\tt AAAA}
queries.

\para{Identifying differences in domain behaviors within a country.} 
We continue using our statistical approach for identifying differences within
a country. We measure the ratio of blocking that occurs for a domain's {\tt A}
and {\tt AAAA} records within the country and then compare these using
a two-proportion $z$-test with a \Sidak corrected $p$-value of $1-.05^{dom}$
where $dom$ is the total number of tested domains (714). We only label the
behavior of a censor with regards to a domain as different over {\tt A} and
{\tt AAAA} queries if the $z$-test finds the difference to be statistically
significant. Once again, we do this to err on the side of caution in order to
minimize over-reporting and false-positives of censorship and, in this case,
its corresponding policy differences over {\tt A} and {\tt AAAA}.

% \para{Which countries have the largest fractions of domains
% exhibiting {\tt A} and {\tt AAAA} resolution inconsistencies?}
% Out of the 714 domains that were tested in each country, Thailand exhibits the greatest
% difference in censorship over {\tt A} and {\tt AAAA} queries. ~30\% of all domains
% tested in Thailand had a signficant difference followed by ~14\% in he US. 8 countries
% below the above 2 in the top 10 exhibit significant difference in censorship over 
% {\tt A} and {\tt AAAA} queries in ~1 to ~5 \% of all domains. 54 countries had no
% domain and 71 countries had atleast 1 domain present which had a significant difference 
% in censorship over {\tt A} and {\tt AAAA} queries.
% 
% \para{Which categories do the domains, with the largest resolution inconsistencies 
% over {\tt A} and {\tt AAAA} queries, belong to?}
% We now analyze the categories of each of the domains which have a significant difference
% in censorship over  {\tt A} and {\tt AAAA} queries. We used the McAfeee URL ticketing system
% \cite{mcafee} to determine categories of the domains that we tested. All our domains fall in
% 61 distinct categories. About ~18\% of domains with a significant difference belonged to the 
% \textit{General News} category and another ~18\% belonged to \textit{Search Engines}. Other 
% popular categories of websites include \textit{Government/Military, Internet Services 
% and Malicious Sites}.

\para{Do these {\tt A/AAAA}-inconsistent domains hint at policy gaps?}
For each country, we begin our analysis by deriving domain categories, using
the McAfee domain categorization service \cite{mcafee}, for domains in the
following two lists: (1) $\text{D}_{\text{any}}$ which contains
the domains which experienced any blocking events inside a country and (2)
$\text{D}_{\text{inconsistent}}$ which contains the {\tt
A/AAAA}-inconsistent domains identified by our $z$-test. Next, we compute the
KL-divergence between the category distributions of the two lists
(\ie $\nabla_{\text{query}}^{\text{domains}}
= \text{KLDivergence}(\text{D}_{\text{any}},
\text{D}_{\text{inconsistent}})$). 
%
A small $\nabla_{\text{query}}^{\text{domains}}$ would signify that the domains
that experience inconsistent treatment are not from a largely different
category distribution than the set of all domains that experience any type of
blocking. This would suggest that the inconsistencies do not arise from
a content-specific policy gap that exists in the censorship mechanism
implemented over {\tt A} and {\tt AAAA} queries. On the other hand, a large
difference would signify that the category distributions are very different and
that domains with specific types of content appear to have more
inconsistencies --- suggesting a content-based policy gap.

Based on this analysis, we find that the United States and Thailand have the
smallest $\nabla_{\text{query}}^{\text{domains}}$ scores (0.4-1.2). Conversely,
Pakistan and Myanmar have high $\nabla_{\text{query}}^{\text{domains}}$
scores (>2). 
%
After manual inspection of these results, we attribute the low divergence in
the United States to the fact that censorship observed in the US arise largely
from a range of corporate networks (\Cref{sec:prevalence}). These resolvers are fairly
consistent in their blocking of domains belonging to McAfee's `P2P/File Sharing' 
,`Malicious Sites' and `Technical/Business Forums' categories. Of the most blocked
individual domains, we see domains related to cryptocurrency (which were categorized
under `Technical/Business Forums'), `netflix.com' and `utorrent.com'. 
%
On the other hand, Thailand shows low divergence despite the large
number of residential networks. This suggests that there is no significant
policy gap that causes the {\tt A/AAAA}-inconsistencies. Rather, it suggests
an incomplete implementation of an existing mechanisms for {\tt A} query
censorship. 
%
On the other hand, in Pakistan and Myanmar where the divergence scores were
high, we found that the biggest contribution to the high divergence scores
arose from the `Pornography' and `Government/Military' domain categories,
respectively. This suggests the presence of a content-policy gap in the
implementation of {\tt A/AAAA} DNS censorship implementations that
disproportionately allows sites in these categories to evade censorship.



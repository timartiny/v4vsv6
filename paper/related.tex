\section{Related Work}\label{sec:related_work}

There is a significant body of previous work investigating DNS based censorship
strategies which can be generally broken down along a few major
fault lines. The first is duration: snapshot studies as compared with longitudinal
measurements. Many initial investigations into censorship techniques and proposals
for measurement methodology provide an evaluation of DNS censorship during a
single snapshot in time~\cite{Anonymous2020:TripletCensors,global2002great,vandersloot2018quack,scott2016satellite,pearce2017global}. Similarly, a snapshot
can capture DNS censorship centering around a specific event like an election or
social uprising~\cite{aryan2013internet}. Established methodology can then be used to
gain perspectives on the way that censorship evolves over time as part of  a
longitudinal measurement~\cite{USESEC21:GFWatch,filasto2012ooni,sundara2020censored,niaki2020iclab}.

The second major fault line that studies split across is scope: studies that focus
in on the DNS censorship strategies used by individual countries as contrasted with global
measurement. Censorship strategies are not universal and each censor is unique
to some degree. Targeted measurement studies contribute to a better
understanding of block-list infrastructure~\cite{ramesh2020decentralized,USESEC21:GFWatch}
and explain blocking phenomena~\cite{global2002great,Anonymous2020:TripletCensors}.
Global studies provide high level view on the use of DNS censorship internationally
providing context and and understanding of prevalence to specific censorship
techniques~\cite{vandersloot2018quack,scott2016satellite,pearce2017global,sundara2020censored,niaki2020iclab}.

% Quack~\cite{vandersloot2018quack}

% satellite~\cite{scott2016satellite},

% Iris: global DNS interference Measurement~\cite{pearce2017global}

% censored planet~\cite{sundara2020censored}

% ICLab: Longitudinal DNS measurement~\cite{niaki2020iclab}

% GFW DNS censorship measurement~\cite{USESEC21:GFWatch}
% % Uses two hosts they control in china and multiple external vantage points to probe

% Triplet Censors: Demystifying GFWs {DNS} Censorship Behavior~\cite{Anonymous2020:TripletCensors}

% OONI

This work describes a methodology for collecting global measurement of IPv6 DNS
censorship data and provides a snapshot analysis of several countries of interest.

\textbf{Using Open Resolvers.}
Open DNS resolvers provide a unique vantage point from which to study the internet
and a significant number of past measurement studies rely on their potentially censored
traversals to measure suspect or adversarial behavior.  Here we present a
non-exhaustive timeline to distinguish methodology and describe primary
contributions of related work relying on open resolvers over last two decades.

In 2007 Lowe et al. queried open resolvers hosted within China
eliciting injected responses to characterize Chinese censorship infrastructure
and strategy~\cite{lowe2007great}. In 2008 Dagon et al.
performed a survey of open DNS resolvers looking for resolvers that would
intentionally provide incorrect or malicious DNS records relating to
phishing attempts~\cite{dagon2008corrupted}.
In 2012 anonymous authors focused in more closely on the explicit censorship of
the GFW DNS injection and the collateral poisoning effect that it had on open
resolvers around the world~\cite{levis2012collateral}. In 2014 Wander et al.
used open resolvers to look more broadly for global poisoning of DNS resolution
by any censoring country finding that spoofed addresses were leaking primarily
from China and Iran ~\cite{wander2014measurement}. Similar to Dagon et al., in 2015
K{\"u}hrer et al. performed a global study of the reliability of open DNS resolution
finding evidence of censorship, injected advertisements, and other suspicious
or malicious behavior by returned addresses~\cite{kuhrer2015going}.

Satellite (2016) outlines a methodology for regularly discovering the set of
available open resolvers and querying hosts in order to detect paths
that return incorrect or inconsistent resource records~\cite{scott2016satellite}.
Iris (2017) relies on a very similar scanning methodology but develops a set of
metrics using follow-up scans and requests that allow the authors to differentiate
between inconsistency, misconfiguration, and manipulation~\cite{pearce2017global}.
These metrics are based
on things like address consistency, TLS certificate validation, HTTP content hash,
geolocation, DNS PTR lookup, and AS information. These supplemental elements
allow Iris to handle cases like CDNs, virtual hosting, distributed / forwarded
resolution requests and more. The 2020 Censored Planet project incorporates
and extends the methods of the Satellite and Iris as part of a comprehensive
and longitudinal global censorship measurement study~\cite{sundara2020censored}.
The measurement platform improves the filtering of open resolvers using a set of
trusted resolvers as a base of truth and validates TLS certificates for returned
resource records. However as with Satellite and Iris, resolvers are discovered
using zmap, limiting the measurement to the IPv4 address space.

\textbf{IPv6 Censorship Measurement.}
Previous censorship studies primarily focus on measurements in IPv4
assuming that the mechanisms and block-lists are largely equivalent or finding
that results were close enough to be considered noise relative to the independent
question at hand. However, several efforts explicitly cover IPv6 results.

In March 2020 Hoang et al. began collection of DNS records injected by the GFW in order to
classify the addresses provided, block-pages injected, and the set of hostnames
that receive injections~\cite{USESEC21:GFWatch}. Their analysis investigates the
commonality of addresses injected by the GFW, finding that all injected
\texttt{AAAA} responses are drawn from the reserved teredo subnet \texttt{2001::/32}.
The longitudinal study does not directly compare the injection rates
of A vs AAAA or differences in injection to DNS queries sent over IPv4 versus IPv6.

A 2021 investigation of HTTP keyword block-lists associated with the GFW
found that results are largely the same between IPv4 and IPv6~\cite{weinberg2021chinese}.
However, the authors note that the GFW failed to apply the temporary 90 second ``penalty box''
blocking subsequent connections between the two hosts described by numerous
previous studies~\cite{xu2011internet,clayton2006ignoring}. This supports our finding
that for now the GFW infrastructure supporting IPv4 and IPv6 are implemented
and/or deployed independently.

\section{Related Work}\label{sec:related}
%% Leaving as TODO for resubmission or revision.... 
% Our work contributes to two domains. First, we make contributions to the
% understanding of DNS censorship, it's implementation in different regions, and
% it's vulnerability to circumvention. Second, we make contributions to the
% understanding of protocol and network inconsistencies that arise due to the
% Internet's transition to IPv6. In this section, we provide an overview of
% research in both domains.
% 
% \subsection{DNS censorship} \label{sec:related:dns}
There is a significant body of previous work investigating DNS based censorship
strategies which can be generally broken down along several dimensions: by their
duration, scope, and vantage points.

\para{Longitudinal or snapshot measurements of DNS censorship.}
Many initial investigations into censorship techniques and
proposals for measurement methodology provide an evaluation of DNS censorship
during a single snapshot in time \cite{Anonymous2020:TripletCensors,
global2002great, vandersloot2018quack, scott2016satellite, pearce2017global}. 
Similarly, a snapshot can capture DNS censorship centering around a specific
event like an election or social uprising \cite{aryan2013internet}.  Established
methodology can then be used to gain perspectives on the way that censorship
evolves over time as part of  a longitudinal measurement
\cite{USESEC21:GFWatch, filasto2012ooni, sundara2020censored, niaki2020iclab,
razaghpanah2016exploring}. Our work provides a snapshot view of DNS censorship
during the period of transition from IPv4 to IPv6.

\para{Regional or global measurements of DNS censorship.}
Censorship strategies are not universal and each censor is unique to some
degree. Targeted measurement studies contribute to a better understanding of
block-list infrastructure ~\cite{ramesh2020decentralized, USESEC21:GFWatch} and
explain blocking phenomena~\cite{global2002great,
Anonymous2020:TripletCensors}. Global studies provide high level view on the
use of DNS censorship internationally providing context and and understanding
of prevalence to specific censorship techniques~\cite{vandersloot2018quack,
scott2016satellite, pearce2017global, sundara2020censored, niaki2020iclab}.
Our work performs a global measurement of DNS censorship, however, through the
lens of the differences in IPv4 and IPv6 censorship deployment, we are able to
make data-driven hypotheses about the censorship infrastructure deployed in
several countries.

\para{Using open resolvers as measurement vantage points.}
Open DNS resolvers provide a unique vantage point from which to study the
Internet and a significant number of previous efforts have relied on them to
gain insights into the censorship mechanisms in different regions.

In 2007, Lowe \etal queried open resolvers hosted within China eliciting
injected responses. This was the first use of open resolvers to characterize
Chinese censorship infrastructure and strategy~\cite{lowe2007great}. In 2008
%Dagon \etal performed a survey of open DNS resolvers looking for resolvers that
%would intentionally provide incorrect or malicious DNS records relating to
%phishing attempts~\cite{dagon2008corrupted}.
In 2012, anonymous authors focused on the Chinese Great Firewall's DNS
injection and the collateral poisoning effect that it had on open resolvers
around the world~\cite{nebuchadnezzar2012collateral}. In 2014, Wander \etal
used open resolvers to look more broadly for global poisoning of DNS resolution
by any censoring country finding that spoofed addresses were leaking primarily
from China and Iran ~\cite{wander2014measurement}. Similar to Dagon \etal, in
2015 K{\"u}hrer \etal performed a global study of the reliability of open DNS
resolution finding evidence of censorship, injected advertisements, and other
suspicious or malicious behavior by returned addresses~\cite{kuhrer2015going}.

More recently, Satellite (2016) outlined a methodology for regularly
discovering the set of available open resolvers and querying hosts in order to
detect paths that return incorrect or inconsistent resource
records~\cite{scott2016satellite}. Iris (2017) relied on a similar scanning
methodology but developed a set of metrics using follow-up scans and requests
that allow the differentiation between inconsistency, misconfiguration, and
manipulation~\cite{pearce2017global}. These metrics are based on several
factors including: address consistency, TLS certificate validation, HTTP
content hash, geolocation, DNS PTR lookup, and AS information. These
supplemental elements allow Iris to handle cases like CDNs, virtual hosting,
distributed / forwarded resolution requests and more. The 2020 Censored Planet
project incorporated and extended the methods of the Satellite and Iris as part
of a comprehensive and longitudinal global censorship measurement
study~\cite{sundara2020censored}. 
%
Although several of these measurement platforms perform filtering to identify
reliable resolvers, they are all limited to using open resolvers in the IPv4
space. Consequently, they are unable to provide insights into the state of IPv6
censorship.

\para{IPv6 censorship measurement.} 
Previous censorship studies primarily focus on measurements in the context of
IPv4. However, there have been several efforts to explicitly incorporate IPv6.
In March 2020 Hoang \etal began collection of DNS records injected by the Great
Firewall in order to classify the addresses provided, block-pages injected, and
the set of hostnames that receive injections~\cite{USESEC21:GFWatch}. Their
analysis investigates the commonality of addresses injected by the GFW, finding
that all injected \texttt{AAAA} responses are drawn from the reserved teredo
subnet \texttt{2001::/32}. However, because this study does not directly
compare the injection rates of A vs AAAA or differences in injection to DNS
queries sent over IPv4 versus IPv6, our efforts complement their findings and
provide a more detailed understanding of IPv6 censorship in China.
%
Although not focused on DNS censorship, a 2021 investigation of HTTP keyword
block-lists associated with the Great Firewall found that results are largely
the same between IPv4 and IPv6~\cite{weinberg2021chinese}. However, the authors
note that over IPv6 connections, the the Firewall failed to apply it's
signature temporary 90 second ``penalty box'' blocking subsequent connections
between the two hosts described by numerous previous
studies~\cite{xu2011internet,clayton2006ignoring}. This supports our finding
that for now the GFW infrastructure supporting IPv4 and IPv6 are implemented
and/or deployed independently.

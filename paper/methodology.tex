\section{Dataset and Methodology}\label{sec:methodology}

The analysis presented in the remainder of this paper is based on the results of
22.4M DNS {\tt A} and {\tt AAAA} resolution requests for 714 domains sent to
7,843 IPv4- and IPv6-capable resolvers located in 128 countries.
%
In this section, we explain our process for identifying resolver targets for
our queries (\Cref{sec:methodology:resolvers}), domains that are the subject of
our queries (\Cref{sec:methodology:domains}), and our process for identifying
the occurrence of a censorship event from the results of each query
(\Cref{sec:methodology:censorship}).

\subsection{Selecting resolvers} \label{sec:methodology:resolvers}
%
Our work is aimed at characterizing the inconsistencies that exist in the
handling of IPv4- and IPv6-related DNS queries --- \ie differences in the
handling of {\tt A} and {\tt AAAA} query types over IPv4 and IPv6
connections. Therefore, we required that each resolver used for our
measurements was IPv4- and IPv6-capable.
%

\para{Identifying resolvers with IPv4- and IPv6-capabilities.}
Our approach, builds on the work of Hendricks \etal
\cite{hendriks2017potential} who identified IPv6 open resolvers to measure the
potential for IPv6-based DDoS attacks.

\parait{IPv6-only domain.}
We begin by creating a new domain, owned and controlled by us, with the name server
set to an IPv6-only record. In other words, to query this domain (which we call
the IPv6-only NS domain), one must contact the IPv6-only name server.
%We begin by creating two new domains, owned and controlled by us and used
%exclusively for this study. The Name Servers for each domain were also
%controlled by us. The Name Server for one of these domains was hosted on
%an IPv6-only network without the capability of communicating with IPv4.
%%
%Note that because these domains were newly registered, they could not have been
%present on the blocklists of any censor. We refer to these domains as our
%{control and and IPv6-only NS domains}.
%

\parait{Identifying IPv4-capable resolvers.}
We use {\tt zmap} \cite{Durumeric13zmap} to scan the entire IPv4 address
space and issue a DNS {\tt A} query on port 53 for a separate IPv4 control
domain we also control. This yields an initial list of 7.2M IPv4 DNS resolvers.
%

\parait{Verifying IPv6 capabilities of IPv4-capable resolvers.}
Next, we issue a DNS {\tt A} query for a resolver-specific subdomain of our
IPv6-only NS domain to each of these IPv4 DNS resolvers. The subdomain encodes
the IPv4 address of the resolver being targeted. Therefore, if our domain was
{\tt v6onlyNS.io} and our resolver target was {\tt 1.1.1.1}, our DNS query
requested the {\tt A} record for {\tt 1-1-1-1.v6onlyNS.io}.
%
Since IPv4-only resolvers will not be able to communicate with our IPv6-only
Name Server, we expect this resolution to fail for IPv4-only resolvers.
On the other hand, resolvers with any form of IPv6 connectivity will be
able to connect to our IPv6-only Name Server.
%
Thus, by examining the logs of our IPv6-only name server, we are able
to identify the set of resolvers that successfully reached our server and
their corresponding IPv4 addresses. The associated IPv6 address for each
successful query is extracted from packet captures of the IPv6-only name
server giving us 33.7K (IPv4, IPv6) address matchings.

\para{Filtering and geolocating resolvers.}
The approach detailed above yields matchings that suggest the presence of
`infrastructure' resolvers --- \eg multiple IPv4 resolvers have the same IPv6
address associated with them. These are cases where the IPv4 resolver simply
forwards requests to a dedicated multi-machine DNS infrastructure rather than
performing the resolution by itself. Although this does not change the validity
of our results regarding the IPv6-related inconsistencies of resolvers, we
still remove these cases in order to minimize the influence of such
infrastructure resolvers. This yields 14.9K resolver pairs.
%
Finally, to confirm the correctness of our list of IPv4/IPv6 resolver pairs, we:
(1) use the Maxmind GeoIP dataset \cite{maxmind-connectiondb} to geolocate the
IPv4 and IPv6 addresses of a resolver pair and keep those which belong to the
same region which reduces us to 14.2K resolver pairs, (2) perform a follow up
scan by issuing {\tt A} and {\tt AAAA} requests for both our control domains
using {\tt zdns} \cite{Durumeric13zmap} and filter out those pairs where an
incorrect response was received.
% 
In total, we obtained 7,843 resolver pairs across 106 different countries.
\Cref{tab:appendix:rates} illustrates the geographic distribution of the
resolvers.


\subsection{Selecting target domains}
\label{sec:methodology:domains}
In order to identify inconsistencies in IPv4/IPv6-related DNS censorship we
require that the domains we use are: (1) sensitive and likely to be censored in
a large number of countries and (2) have valid {\tt A} and {\tt AAAA} records
associated with them.

\para{Identifying sensitive domains.} Censored Planet's `Satellite' project,
which performs global longitudinal measurements of DNS censorship,
\cite{sundara2020censored} maintains a list of domains that combines sensitive
domains in each country with popular domains randomly chosen from the Alexa
Top-10K. The sensitive domains in this list were gathered by the Citizen Lab
using regional experts to curate lists for each country
\cite{citizenlab-blocklists}. Given the input from experts and the availability
of comparable validation data from Censored Planet, we utilize this list as the
starting point for our study as well. We start with 2506 sensitive domains.

\para{Identifying usable domains.} Not all the domains on the Satellite list
are usable in our study since they do not have IPv6 connectivity or {\tt AAAA}
records.
%
We filter out unusable domains by using {\tt zdns} to perform {\tt A} and {\tt
AAAA} resource record requests from Google and Cloudflare's four public DNS
resolvers ({\tt 8.8.8.8}, {\tt 8.8.4.4}, {\tt 1.1.1.1}, and {\tt 1.0.0.1}) and
removing those with invalid {\tt A} or {\tt AAAA} records. A record is invalid
if it does not contain valid resource appropriate IPv4 or IPv6 addresses. This
gives us with 775 sensitive and infrastructure appropriate domains.
%
Finally, we follow-up by making TLS connections (using {\tt zgrab2}), from our
uncensored vantage point, to each of the IP addresses contained in the DNS
responses. We locally verify the obtained TLS certificates and exclude all
domains whose verification fails. 
%
This final list contains 714 sensitive domains whose TLS certificates and IPv4
and IPv6 addresses are valid. We use this list for all the measurements
reported in this paper.

\subsection{Identifying DNS censorship events} \label{sec:methodology:censorship}
In general, our goal is to err on the side of caution and avoid false-positives
in our censorship determination. We achieve this by accounting for
unreliability of resolvers and domains in our lists.

\para{Removing unstable resolvers.}
For each of the 7,843 (IPv4, IPv6) resolver pairs and 714 sensitive domains, we
send a DNS {\tt A} and {\tt AAAA} resource record request for a domain to the
IPv4 and IPv6 addresses associated with the resolver. We follow this up with
a {\tt A} and {\tt AAAA} resource record request for a set of 3 control domains
(owned and operated by us).
%
We discard data from pairs which failed to resolve any one of our control
domains correctly since this is a sign of resolver instability. This left us
with 7,441 stable resolver pairs.
%

\para{Distinguishing censorship from domain instability.}
For the remaining resolvers, we extract the IPv4 and IPv6 resource records
returned for each domain --- even those that arrive from multiple responses to
a single query (a sign of an on-path censor). Next, we use \texttt{zgrab2} to
establish TLS connections to the IP addresses contained in the resource
records. In each of these connections, we set the TLS SNI to be the domain
whose records were requested. We then locally verify the validity of the
retrieved TLS certificates.
%
Since the domains themselves might be unreliable, we repeat this verification
procedure three times. Only if this step fails all three times do we conclude
that the IP we obtained from that resolver for that domain was censored. 

\subsection{Ethics}\label{sec:methodology:ethics}
Our experimental design has incorporated ethical considerations into the
decision-making process at multiple stages.
%We concern ourselves primarily with
%the security and autonomy of human users and their ability to securely access
%the internet. 
%Given the prevalence of self-censorship, understanding and
%documenting the mechanisms and scope of censorship strategies can assist users
%in better evaluating their own risk. However, censorship measurement naturally
%entails interacting with and occasionally violating the rules of access control
%systems. 
Censorship measurement has inherent risks and trade-offs: better understanding
of censorship can help support and inform users, but specific measurements may
carry risk to participants or network users.
We rely on The Menlo Report~\cite{menlo}, its companion
guide~\cite{menlo-companion}, and the censorship specific ethical measurement
guidelines discussed by Jones \etal \cite{jones2015ethical} to 
carefully weigh these trade-offs in our experimental design.
%provide structure and guidance to our experimental design.

\para{Consent.}
To align with the guiding principle of \textit{respect for persons} we
structure the data collection to implicate as few individuals as possible.
Specifically we rely on open resolvers which typically have little or no direct
association with individuals in lieu of measurement from client based
software. While we cannot acquire direct or proxy consent from the operators of
the open resolvers we consider the trade-off between the implied consent
standard and the value in the measurements we make. 
%Understanding the extent
%and mechanisms of censorship infrastructure can help to demystify and clarify
%risks to real users seeking to safely access the internet. It is important to
We note that the goal of our ethical analysis is not to eliminate risk, but to
minimize it wherever possible. As noted by Jones \etal in some cases acquiring
consent from operators may not only be impossible, but could increase the risk to
operators as it introduces their acknowledgement of, or active participation
in, the measurement at hand~\cite{jones2015ethical}. This analysis aligns with
previous work relying on open resolvers to collect impactful results while
minimizing risk on
individuals~\cite{pearce2017global,scott2016satellite,sundara2020censored}.

\para{Privacy.}
Our study collects no personal data about any end~users or open
resolver operators. The analysis completed herein uses resolver addresses,
public Anonymous System (AS) identifiers, and country codes.  All measurements
are initiated from within the United States.
%at little to no risk of
%repercussions to citizens in the countries that we examine.
Beyond this,
measurement domains are not drawn from any human browsing patterns or history
as the suspected censored domains are a subset of the Satellite measurement
results (\cf \Cref{sec:methodology:domains}).

\para{Resource usage.} 
The vantage that was used for data collection is connected to the internet with
a 1 Gbps interface that scanned using the default rates for {\tt zmap} and {\tt
zgrab2} tools (line rate). However, the structure of the scan was established
such that individual resolvers and domains for the DNS probe and TLS
certificate validation respectively would be accessed in round robin order ---
\ie when probing the open resolvers every target would receive a first
request before any target would receive the subsequent request.  Equivalently
for validating TLS certificates, each of the 714 target domains would receive
a first attempted handshake before any target would receive a subsequent
handshake, limiting the bandwidth any one host will receive.
%While we do not have an upper bound on the bandwidth that was sent
%to individual resolvers or TLS endpoints we believe this strategy provides
%a reasonable limitation to the impact of the measurements in this study.


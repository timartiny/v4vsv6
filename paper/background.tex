\section{Background}\label{sec:background}
Our work is focused on uncovering differences in {\em censorship} that occur
from {\em DNS censorship} mechanisms' failure to effectively adapt to the Internet's
{\em adoption of the IPv6 protocol}. 
%
In this section, we provide a working definition of censorship, an overview of
DNS censorship implementations, and an overview of parts of the IPv6 protocol
that are relevant to DNS censorship mechanisms and their measurement.

\subsection{Internet censorship}\label{sec:background:censorship}

\para{What constitutes Internet censorship?}
Internet censorship can be broadly defined as the act of filtering or blocking
access to Internet content \cite{townsend}. Like previous work, we also use
this definition which applies to any Internet traffic filtering or blocking
actions, regardless of the entities that perform them --- \ie they may be
corporate network firewalls, ISPs, or national censorship apparatus.

\para{Distinguishing between corporate and national censorship.}
Although we use a broad definition of censorship, throughout our work, we
take care to identify when it is likely that our measurements are reporting
instances of corporate censorship rather than national censorship. 
This classification is done with the use of Maxmind's
Connection Type database \cite{maxmind-connectiondb} which characterizes IP
end-points as `Corporate', `Cable/DSL' (associated typically with residential
networks), or `Cellular'.
%
This is important in order to provide context for some of our results.
For example, from our study we find that 0.9\% of our tests in the United
States were censored. Without the additional context that: (1) more that 50\%
of all our measurement vantage points in the US were placed in corporate
networks and (2) these end-points were responsible for the identified
censorship events, this finding would be confusing or, worse, assumed
incorrect. 

\para{Internet censorship mechanisms.} Blocking access to content can occur in
a variety of ways. 

\subsection{DNS censorship}\label{sec:background:dns}


Network based censorship is a common barrier to global internet access today.
% \FIXME{Corporate censorship}
In some instances this manifests as corporate access control mechanisms deployed
to protect internal networks and intellectual property. However,
state actors have the ability to incorporate large scale network monitoring
systems into national network infrastructure preventing access to information
through either active interference or threat of retribution. There
have been many studies on censorship techniques both
globally~\cite{pearce2017global,niaki2020iclab,scott2016satellite,sundara2020censored,filasto2012ooni,pearce2017augur}
and within individual countries~\cite{USESEC21:GFWatch,aryan2013internet,ramesh2020decentralized,yadav2018light,gebhart2017internet,nabi2013anatomy}.

Internet infrastructure is evolving to accommodate global connectivity as well
-- IPv6 deployment provides routable addressing to an increasing portion of the
internet as IPv4 allocations become scarce. Effective censorship strategies
attempting to control or limit accessibility have to keep up.
Censorship strategies are capable of reactive change as demonstrated by
the impact that real world events like elections~\cite{aryan2013internet},
social or political unrest~\cite{shandler2018measuring,padmanabhan2021multi},
and specifically crafted circumvention tools~\cite{beznazwy2020china}
have on blocking techniques and block-lists. Documenting contemporary censorship
strategies through the transition from IPv4 to IPv6 can help to understand the
the trajectory of network censorship efforts more broadly.


\subsection{DNS Censorship Measurement}

\FigDNSCensorship

The Domain Name System (DNS) underpins the global internet by providing a
mapping from human readable hostnames to routable IP addresses making domain name
resolution the first step in almost all connection establishment flows. However,
the widely deployed DNS system is implemented as a plaintext protocol allowing
on-path eavesdroppers to inspect the hostnames as clients attempt to establish
connections and in some cases inject falsified responses to interfere.

The Chinese traffic inspection system, called the Great Firewall (GFW),
is documented injecting falsified DNS responses as early as 2002~\cite{global2002great}.
This censorship has been shown to be a packet injection from an on-path adversary
monitoring for hostnames in DNS queries that match regular
expressions~\cite{USESEC21:GFWatch}. An on-path adversary
operates on a copy of traffic transiting a specific link or gateway in a network.
This contrasts with an in-path adversary which operates on traffic inline with
the capability to drop or modify packets on the fly. These traffic inspection
devices can be housed at or near border gateways~\cite{xu2011internet} or
distributed throughout regional ISPs~\cite{ramesh2020decentralized} within
censoring countries. Notably the GFW and other on-path DNS censorship appliances
will inject responses in either direction (downstream into the country or
upstream out out of the country) to the source of the query. This allows
measurements of deployed censorship infrastructure to be launched from outside
of the country in question as long as the query transits a monitored link.

Common strategies for measuring censorship via DNS injection involve routing DNS queries for
block-listed hostnames across a censoring link in a controlled environment where
the returned resource records can be independently validated. Two notable previous
studies Satellite and Iris~\cite{scott2016satellite,pearce2017global} identify
open DNS resolvers across the internet by scanning the entire IPv4 space on port~53.
Satellite probes reliably available open resolvers from a set of distributed vantage
points and detects incorrect or inconsistent response information. Iris develops
a method for identifying active manipulation in contrast to misconfiguration by
leveraging metrics such as consistency and independent verifiability. Iris
compares records returned by open resolvers to records returned a set of trusted
resolvers matching IP addresses, content hash, TLS certificate, and more.
In order to identify open resolvers both Satellite and Iris rely on the zmap
internet wide scanning tool~\cite{Durumeric13zmap} limiting themselves to the IPv4 address space.

\subsection{IPv6 \& DNS}
\label{subsec:v4vsv6}

The proportion of clients that support IPv6 is rapidly growing, especially in
developing areas with newly deployed network infrastructure.
According to the APNIC internet registry over a quarter of the users
on the internet now route their traffic using IPv6~\cite{Huston-APNIC2021}.
Similarly Google metrics indicate that over 50\% of users access services using
IPv6 in India, Saudi Arabia, Germany and several other countries~\cite{Google-IPv6}.
As with many other critical internet protocols, DNS implements an IPv6 extension
to maintain relative uniformity in connection establishment flows.

\textbf{A vs AAAA.} DNS \texttt{A} queries and records respectively request and provide
resources to resolve hostnames into IPv4 addresses. \texttt{AAAA} queries and
records resolve hostnames to IPv6 addresses.

\textbf{IPv4 vs IPv6 Queries.}The type of resource requested is
indicated in the \texttt{query\_type} portion of a DNS request and is not linked
to the IP version that the DNS query is sent over - i.e. both \texttt{A} and
\texttt{AAAA} can be resolved over IPv4 or IPv6.
Some hostnames resolve exclusively to IPv4 or IPv6 i.e. only provide \texttt{A}
or \texttt{AAAA} records respectively, as is the case for many legacy sites on
the internet that only support IPv4.

A theoretically comprehensive DNS censorship strategy using response injection
requires traffic monitoring infrastructure to analyze both IPv4 and IPv6
traffic and parse both \texttt{A} and \texttt{AAAA} queries accounting for hostnames
that may not implement resource records of one type or the other. Such a strategy
would only impede connections that use plaintext DNS (as outlined in the
original specifications~\cite{RFC1035,RFC3596}) - it would not effect more secure DNS
protocols such as DoT, DoH, or censorship resistance strategies which are not
covered in this work.

%% Are there appliances that inject responses to AAAA requests upon seeing a
%% block-listed domain name where the actual site doesn't support IPv6 - and
%% should never provide AAAA records in response?
\textbf{Discovering Dual-Stack Open Resolvers.}
Hendricks et al. describe in depth a method for discovering and validating DNS
resolvers that expose both IPv4 and IPv6 interfaces~\cite{hendriks2017potential}.
They start with an initial
set of IPv4 resolvers discovered using zmap. This set is then sent a query for a
specially crafted domain containing a subdomain whose nameserver supports only
IPv6. In an attempt to recursively resolve the domain the resolver will connect
to the controlled nameserver using an IPv6 address if one is available.
Subsequent queries are then used to identify configurations that forward or
distribute queries indicating address pairs that are not truly dual-stack resolvers.

Our work combines these strategies to investigate the global prevalence of DNS
injection in IPv6.

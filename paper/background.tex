\section{Background}\label{sec:background}
Our work is focused on uncovering differences in {\em Internet censorship} that occur
from {\em DNS censorship} mechanisms' failure to effectively adapt to the Internet's
{\em adoption of the IPv6 protocol}. In this section, we provide an overview of
Internet censorship approaches (\Cref{sec:background:censorship}), DNS
censorship mechanisms and infrastructure (\Cref{sec:background:dns}), and the
impact of IPv6 on DNS censorship (\Cref{sec:background:ipv6}).

\subsection{Internet censorship}\label{sec:background:censorship}

\para{What constitutes Internet censorship?}
Internet censorship can be broadly defined as the act of filtering or blocking
access to Internet content \cite{townsend}. Like previous work, we also use
this definition when measuring censorship.

\para{Corporate and national censorship.}
Our working definition of censorship applies to any Internet traffic filtering
or blocking actions, regardless of the entities that perform them or their purpose for doing so. With a few
exceptions, much of prior work has largely focused on measuring censorship with
a presumed attribution of measured events to national content access policies.
However, it is important to note that corporate censorship, where businesses
deploy information controls systems to prevent access to specific types of
content, is also globally prevalent. 
%
Throughout our work, we take care to identify when it is likely that our
measurements are reporting instances of corporate censorship rather than
national censorship.  This classification is done using Maxmind's Connection
Type database \cite{maxmind-connectiondb} which characterizes IP end-points as
`Corporate', `Cable/DSL' (associated typically with residential networks), or
`Cellular'.
%
This distinction is important in order to provide context for some of our
results. For example, from our study we find that 0.9\% of our tests in the
United States were censored. Without the additional context that: (1) more that
50\% of all our measurement vantage points in the US were placed in corporate
networks and (2) these end-points were responsible for the identified
censorship events, this finding would be confusing, assumed incorrect, or
incorrectly attributed to a national censorship policy. 

\para{Internet censorship techniques.} 
%
Blocking access to Internet content can occur in a variety of ways and at
different layers of the networking stack, all of which require the ability to
monitor network traffic passing through the censor's borders. The simplest and
most common approaches are: (1) IP-based blocking in which a censor maintains
blocklists of IP addresses and prevent connections to these IP addresses; (2)
DNS manipulation where censors inject (when the censor is a man-in-the-middle)
or return (when the censor is the resolver) incorrect responses to domains that
are to be censored; and (3) HTTP proxying in which a censor acts as a proxy to
clients within its border with the intention of blocking access to intercepted
`unsuitable' content. 
%
While these censorship mechanisms are passive and possible to evade, it is
known that countries such as China have deployed comprehensive censorship
infrastructure that is capable of active probing, protocol inspection, and
incorporates multiple approaches for censorship.
%
In fact, there has been a large body of work to identify censorship techniques
globally
\cite{pearce2017global, niaki2020iclab, scott2016satellite,
sundara2020censored, filasto2012ooni, pearce2017augur, razaghpanah2016exploring}
and specific to individual countries \cite{USESEC21:GFWatch, aryan2013internet,
ramesh2020decentralized, yadav2018light, gebhart2017internet, nabi2013anatomy}.
%
Unfortunately, the impact of the growing deployment of IPv6 networks has not
been previously studied --- leaving gaps in our knowledge of how censors are
handling the network transition and what opportunities exist for developers of
circumvention tools. Our work fills this gap by studying how the transition to
IPv6 impacts DNS censorship mechanisms.

\subsection{DNS censorship}\label{sec:background:dns}

% removing figure for now... needs work
%\FigDNSCensorship

The Domain Name System (DNS) underpins the global internet by providing
a mapping from human readable hostnames to routable IP addresses making domain
name resolution the first step in almost all connection establishment flows.
However, the widely deployed DNS system is implemented as a plaintext protocol
allowing on-path eavesdroppers to inspect the hostnames as clients attempt to
establish connections and in some cases inject falsified responses to interfere.


Censors have long been known to use DNS injection to block requests, observing
DNS requests and injecting false responses for requests to censored domains.
The Chinese traffic inspection system, called the Great Firewall (GFW), is
documented injecting falsified DNS responses as early as
2002~\cite{global2002great}. This censorship has been shown to be a packet
injection from an on-path adversary monitoring for hostnames in DNS queries
that match regular expressions~\cite{USESEC21:GFWatch}.
%Other countries are also observed to perform DNS injection

%\para{Directionality}
%An on-path censor can observe DNS requests for blocked domains and inject a false
%response, providing an incorrect answer that beats the correct answer. While
%censors typically do this on the edge of their
%network~\cite{xu2011internet} with the goal of blocking requests for users
%within the country, these systems often work \emph{bidirectionally}, blocking
%even requests that originate outside the country.
%
%However, even if a censor manages to only censor requests from inside the
%country to the outside, the nature of DNS resolvers allow it to be externally
%visible. For instance, we can make a request to an in-country resolver, which
%will in turn send a DNS request to the authoritative nameserver which may be
%outside the country. If this request is censored

%\para{On-path and in-path DNS censorship.}
%An on-path censor operates on a copy of the traffic that transits a specific
%network link. Operating on a copy allows the censor to make deeper inspections
%on the content of packets and respond accordingly. This approach has the
%limitation that censors cannot interfere with the packets that have already
%passed their vantage point. In the context of DNS censorship, on-path censors
%make censorship decisions based on the copy of a DNS request and inject a false
%response to the source of the request in the hopes that their response reaches
%the source before the actual response from a resolver. Because the source will
%receive a response from the intended resolver as well as the injected response
%from the censor, on-path censorship can be identified by recording {\em all}
%DNS response packets that are received to a DNS query for a censored domain.
%%
%Another censorship approach is in-path censorship in which a censor makes
%decisions about dropping or modifying packets on-the-fly and at line rate.
%Censoring of DNS queries by in-path censors can be identified either by the
%presence of timeouts (packet drops by censor) or incorrect responses (packet
%modifications by censors).
%

\para{Distributed and centralized DNS censorship.}
In a centralized DNS censorship mechanism, a single entity manages the
`blocklists' or filter-rules associated with DNS censorship decisions. This is
the case in countries such as China where traffic inspection devices are housed
at or near border gateways \cite{xu2011internet}. In contrast, countries such
as India and Russia are known to delegate censorship orders to regional
ISPs~\cite{Gosain2017a}
who may choose to implement them either via DNS (typically by reconfiguring
their own resolvers) or other censorship approaches
\cite{ramesh2020decentralized, Yadav2018a, singh2020india}. The latter approach results in
different implementations of DNS censorship and the possibility for
inconsistencies across different ISPs.

% Common strategies for measuring censorship via DNS injection involve routing DNS queries for
% block-listed hostnames across a censoring link in a controlled environment where
% the returned resource records can be independently validated. Two notable previous
% studies Satellite and Iris~\cite{scott2016satellite,pearce2017global} identify
% open DNS resolvers across the internet by scanning the entire IPv4 space on port~53.
% Satellite probes reliably available open resolvers from a set of distributed vantage
% points and detects incorrect or inconsistent response information. Iris develops
% a method for identifying active manipulation in contrast to misconfiguration by
% leveraging metrics such as consistency and independent verifiability. Iris
% compares records returned by open resolvers to records returned a set of trusted
% resolvers matching IP addresses, content hash, TLS certificate, and more.
% In order to identify open resolvers both Satellite and Iris rely on the zmap
% internet wide scanning tool~\cite{Durumeric13zmap} limiting themselves to the IPv4 address space.
% 
\subsection{IPv6 and DNS} \label{sec:background:ipv6}
% The proportion of clients that support IPv6 is rapidly growing, especially in
% developing areas with newly deployed network infrastructure. According to the
% APNIC internet registry over a quarter of the users on the internet now route
% their traffic using IPv6~\cite{Huston-APNIC2021}. Similarly Google metrics
% indicate that over 50\% of users access services using IPv6 in India, Saudi
% Arabia, Germany and several other countries~\cite{Google-IPv6}. 

As with other critical protocols, DNS has also adapted to the IPv6 protocol.
The {\tt AAAA} resource record type was introduced to aid in the resolution of
domains to their IPv6 addresses. Further, the existing DNS protocol is
IP-independent and can therefore be deployed on IPv4 and IPv6 networks.
Therefore, it is now possible and common for IPv4 and IPv6-hosted DNS servers
to receive both {\tt A} and {\tt AAAA} queries. This is in contrast to
a IPv4-dominant Internet where {\tt A} queries to IPv4 resolvers were the norm.

\para{Impact of IPv6 on DNS censorship.}
The changes outlined above also influence the mechanics and success of DNS
censorship operations. A theoretically comprehensive DNS censorship strategy
using response injection requires traffic monitoring infrastructure to: (1)
analyze both IPv4 and IPv6 traffic and (2) parse both \texttt{A} and
\texttt{AAAA} queries accounting for hostnames that may not implement resource
records of one type or the other. 
%
Our work provides a snapshot documentation of contemporary
censorship strategies through the IPv6 transition.

%In addition to helping
%circumvention tool developers exploit weaknesses made available through this
%transition, this work also improves our understanding of the reactiveness of
%network censorship more broadly.
% 
% Such a strategy would only impede connections that use plaintext DNS (as outlined in the 
% original specifications~\cite{RFC1035,RFC3596}) - it would not effect more secure DNS
% protocols such as DoT, DoH, or censorship resistance strategies which are not
% covered in this work.
% 
% Internet infrastructure is evolving to accommodate global connectivity as well
% -- IPv6 deployment provides routable addressing to an increasing portion of
% the internet as IPv4 allocations become scarce. Effective censorship strategies
% attempting to control or limit accessibility have to keep up.
% Censorship strategies are capable of reactive change as demonstrated by
% the impact that real world events like elections~\cite{aryan2013internet},
% social or political unrest~\cite{shandler2018measuring,padmanabhan2021multi},
% and specifically crafted circumvention tools~\cite{beznazwy2020china}
% have on blocking techniques and block-lists. Documenting contemporary censorship
% strategies through the transition from IPv4 to IPv6 can help to understand the
% the trajectory of network censorship efforts more broadly.
% 
% %% Are there appliances that inject responses to AAAA requests upon seeing a
% %% block-listed domain name where the actual site doesn't support IPv6 - and
% %% should never provide AAAA records in response?
% \textbf{Discovering Dual-Stack Open Resolvers.}
% Hendricks et al. describe in depth a method for discovering and validating DNS
% resolvers that expose both IPv4 and IPv6 interfaces~\cite{hendriks2017potential}.
% They start with an initial
% set of IPv4 resolvers discovered using zmap. This set is then sent a query for a
% specially crafted domain containing a subdomain whose nameserver supports only
% IPv6. In an attempt to recursively resolve the domain the resolver will connect
% to the controlled nameserver using an IPv6 address if one is available.
% Subsequent queries are then used to identify configurations that forward or
% distribute queries indicating address pairs that are not truly dual-stack resolvers.
% 
% Our work combines these strategies to investigate the global prevalence of DNS
% injection in IPv6.

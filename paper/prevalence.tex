\section{Prevalence of DNS Censorship}
\label{sec:prevalence}

\para{Overview.} In this section, we focus on \emph{providing a high-level
understanding on the prevalence of DNS censorship on IPv4 and IPv6 networks}.
Specifically, we measure the global prevalence of DNS censorship that occur in
the following four cases: 
%
(1) a DNS {\tt A} query is sent over IPv4, 
(2) a DNS {\tt AAAA} query is sent over IPv4, 
(3) a DNS {\tt A} query is sent over IPv6, and 
(4) a DNS {\tt AAAA} query is sent over IPv6.
%
While much of prior work has focused on case (1), the increased adoption of
IPv6 necessitates the analysis of cases (2-4) which are provided in our work.
%
In each case, we use our collected dataset (\cf \Cref{sec:methodology}) to
summarize the base rate of censorship.

\begin{table}[t]
  \centering
  \small
  \scalebox{\tabularscale}{
  \begin{tabular}{lcccccc}
    \toprule
    {\bf Country} &  {\bf Resolver} & {\bf IPv4} & {\bf IPv4} & {\bf IPv6} & {\bf IPv6} & {\bf Avg.} \\
    {}  & {\bf pairs} & {\tt A} & {\tt AAAA}  & {\tt A} & {\tt AAAA} & {}\\
    \midrule
    China (CN) & \color{black} 194 & {\cellcolor[HTML]{E1EDF8}}
    \color[HTML]{000000} \color{black} 29.27 & {\cellcolor[HTML]{6CAED6}}
    \color[HTML]{F1F1F1} \color{black} 32.32 & {\cellcolor[HTML]{F7FBFF}}
    \color[HTML]{000000} \color{black} 28.41 & {\cellcolor[HTML]{77B5D9}}
    \color[HTML]{000000} \color{black} 32.08 & \color{black} 30.52 \\
    Iran (IR) & \color{black} 277 & {\cellcolor[HTML]{6AAED6}}
    \color[HTML]{F1F1F1}
    \color{black} 25.12 & {\cellcolor[HTML]{8DC1DD}} \color[HTML]{000000}
    \color{black} 24.49 & {\cellcolor[HTML]{EAF2FB}} \color[HTML]{000000}
    \color{black} 21.95 & {\cellcolor[HTML]{F7FBFF}} \color[HTML]{000000}
    \color{black} 21.45 & \color{black} 23.25 \\
    Hong Kong (HK) & \color{black} 67 & {\cellcolor[HTML]{6CAED6}}
    \color[HTML]{F1F1F1} \color{black} 7.00 & {\cellcolor[HTML]{D8E7F5}}
    \color[HTML]{000000} \color{black} 5.57 & {\cellcolor[HTML]{F7FBFF}}
    \color[HTML]{000000} \color{black} 4.91 & {\cellcolor[HTML]{E7F1FA}}
    \color[HTML]{000000} \color{black} 5.24 & \color{black} 5.68 \\
    Russia (RU) & \color{black} 312 & {\cellcolor[HTML]{6AAED6}}
    \color[HTML]{F1F1F1} \color{black} 5.51 & {\cellcolor[HTML]{DCE9F6}}
    \color[HTML]{000000} \color{black} 4.78 & {\cellcolor[HTML]{F7FBFF}}
    \color[HTML]{000000} \color{black} 4.49 & {\cellcolor[HTML]{F7FBFF}}
    \color[HTML]{000000} \color{black} 4.50 & \color{black} 4.82 \\
    Ukraine (UA) & \color{black} 35 & {\cellcolor[HTML]{6CAED6}}
    \color[HTML]{F1F1F1} \color{black} 6.49 & {\cellcolor[HTML]{EDF4FC}}
    \color[HTML]{000000} \color{black} 3.05 & {\cellcolor[HTML]{F7FBFF}}
    \color[HTML]{000000} \color{black} 2.64 & {\cellcolor[HTML]{F2F7FD}}
    \color[HTML]{000000} \color{black} 2.87 & \color{black} 3.76 \\
    Indonesia (ID) & \color{black} 56 & {\cellcolor[HTML]{6AAED6}}
    \color[HTML]{F1F1F1} \color{black} 6.46 & {\cellcolor[HTML]{E6F0F9}}
    \color[HTML]{000000} \color{black} 2.99 & {\cellcolor[HTML]{F4F9FE}}
    \color[HTML]{000000} \color{black} 2.41 & {\cellcolor[HTML]{F7FBFF}}
    \color[HTML]{000000} \color{black} 2.26 & \color{black} 3.53 \\
    Argentina (AR) & \color{black} 47 & {\cellcolor[HTML]{6CAED6}}
    \color[HTML]{F1F1F1} \color{black} 5.20 & {\cellcolor[HTML]{BCD7EB}}
    \color[HTML]{000000} \color{black} 3.51 & {\cellcolor[HTML]{DFECF7}}
    \color[HTML]{000000} \color{black} 2.22 & {\cellcolor[HTML]{F7FBFF}}
    \color[HTML]{000000} \color{black} 1.28 & \color{black} 3.05 \\
    Thailand (TH) & \color{black} 186 & {\cellcolor[HTML]{6AAED6}}
    \color[HTML]{F1F1F1} \color{black} 8.25 & {\cellcolor[HTML]{F4F9FE}}
    \color[HTML]{000000} \color{black} 1.18 & {\cellcolor[HTML]{F5F9FE}}
    \color[HTML]{000000} \color{black} 1.13 & {\cellcolor[HTML]{F7FBFF}}
    \color[HTML]{000000} \color{black} 0.93 & \color{black} 2.87 \\
    Malaysia (MY) & \color{black} 50 & {\cellcolor[HTML]{6AAED6}}
    \color[HTML]{F1F1F1} \color{black} 4.92 & {\cellcolor[HTML]{E3EEF8}}
    \color[HTML]{000000} \color{black} 2.03 & {\cellcolor[HTML]{F7FBFF}}
    \color[HTML]{000000} \color{black} 1.27 & {\cellcolor[HTML]{F5FAFE}}
    \color[HTML]{000000} \color{black} 1.35 & \color{black} 2.39 \\
    Mexico (MX) & \color{black} 150 & {\cellcolor[HTML]{6AAED6}}
    \color[HTML]{F1F1F1} \color{black} 3.78 & {\cellcolor[HTML]{E9F2FA}}
    \color[HTML]{000000} \color{black} 2.05 & {\cellcolor[HTML]{F7FBFF}}
    \color[HTML]{000000} \color{black} 1.75 & {\cellcolor[HTML]{EEF5FC}}
    \color[HTML]{000000} \color{black} 1.94 & \color{black} 2.38 \\
    Bangladesh (BD) & \color{black} 29 & {\cellcolor[HTML]{6AAED6}}
    \color[HTML]{F1F1F1} \color{black} 6.42 & {\cellcolor[HTML]{EFF6FC}}
    \color[HTML]{000000} \color{black} 1.28 & {\cellcolor[HTML]{F6FAFF}}
    \color[HTML]{000000} \color{black} 0.89 & {\cellcolor[HTML]{F7FBFF}}
    \color[HTML]{000000} \color{black} 0.81 & \color{black} 2.35 \\
    Colombia (CO) & \color{black} 27 & {\cellcolor[HTML]{6AAED6}}
    \color[HTML]{F1F1F1} \color{black} 3.39 & {\cellcolor[HTML]{75B4D8}}
    \color[HTML]{000000} \color{black} 3.28 & {\cellcolor[HTML]{EAF2FB}}
    \color[HTML]{000000} \color{black} 1.45 & {\cellcolor[HTML]{F7FBFF}}
    \color[HTML]{000000} \color{black} 1.14 & \color{black} 2.31 \\
    Italy (IT) & \color{black} 38 & {\cellcolor[HTML]{B7D4EA}}
    \color[HTML]{000000}
    \color{black} 2.36 & {\cellcolor[HTML]{F7FBFF}} \color[HTML]{000000}
    \color{black} 2.10 & {\cellcolor[HTML]{D6E5F4}} \color[HTML]{000000}
    \color{black} 2.25 & {\cellcolor[HTML]{6AAED6}} \color[HTML]{F1F1F1}
    \color{black} 2.52 & \color{black} 2.31 \\
    Brazil (BR) & \color{black} 160 & {\cellcolor[HTML]{6AAED6}}
    \color[HTML]{F1F1F1} \color{black} 2.67 & {\cellcolor[HTML]{F7FBFF}}
    \color[HTML]{000000} \color{black} 1.67 & {\cellcolor[HTML]{D8E7F5}}
    \color[HTML]{000000} \color{black} 1.99 & {\cellcolor[HTML]{CFE1F2}}
    \color[HTML]{000000} \color{black} 2.08 & \color{black} 2.10 \\
    Bulgaria (BG) & \color{black} 30 & {\cellcolor[HTML]{6AAED6}}
    \color[HTML]{F1F1F1} \color{black} 3.24 & {\cellcolor[HTML]{89BEDC}}
    \color[HTML]{000000} \color{black} 2.91 & {\cellcolor[HTML]{F4F9FE}}
    \color[HTML]{000000} \color{black} 1.15 & {\cellcolor[HTML]{F7FBFF}}
    \color[HTML]{000000} \color{black} 1.06 & \color{black} 2.09 \\
    Poland (PL) & \color{black} 48 & {\cellcolor[HTML]{C1D9ED}}
    \color[HTML]{000000} \color{black} 2.00 & {\cellcolor[HTML]{EEF5FC}}
    \color[HTML]{000000} \color{black} 1.13 & {\cellcolor[HTML]{6AAED6}}
    \color[HTML]{F1F1F1} \color{black} 2.90 & {\cellcolor[HTML]{F7FBFF}}
    \color[HTML]{000000} \color{black} 0.96 & \color{black} 1.75 \\
    South Africa (ZA) & \color{black} 93 & {\cellcolor[HTML]{6AAED6}}
    \color[HTML]{F1F1F1} \color{black} 2.41 & {\cellcolor[HTML]{D5E5F4}}
    \color[HTML]{000000} \color{black} 1.60 & {\cellcolor[HTML]{F7FBFF}}
    \color[HTML]{000000} \color{black} 1.18 & {\cellcolor[HTML]{EBF3FB}}
    \color[HTML]{000000} \color{black} 1.33 & \color{black} 1.63 \\
    Korea (KR) & \color{black} 632 & {\cellcolor[HTML]{6AAED6}}
    \color[HTML]{F1F1F1} \color{black} 2.33 & {\cellcolor[HTML]{F7FBFF}}
    \color[HTML]{000000} \color{black} 1.06 & {\cellcolor[HTML]{E8F1FA}}
    \color[HTML]{000000} \color{black} 1.24 & {\cellcolor[HTML]{EAF3FB}}
    \color[HTML]{000000} \color{black} 1.22 & \color{black} 1.46 \\
    Chile (CL) & \color{black} 65 & {\cellcolor[HTML]{6CAED6}}
    \color[HTML]{F1F1F1}
    \color{black} 2.67 & {\cellcolor[HTML]{F7FBFF}} \color[HTML]{000000}
    \color{black} 0.70 & {\cellcolor[HTML]{E3EEF9}} \color[HTML]{000000}
    \color{black} 1.10 & {\cellcolor[HTML]{F3F8FE}} \color[HTML]{000000}
    \color{black} 0.79 & \color{black} 1.32 \\
    Romania (RO) & \color{black} 44 & {\cellcolor[HTML]{6AAED6}}
    \color[HTML]{F1F1F1} \color{black} 2.30 & {\cellcolor[HTML]{EAF2FB}}
    \color[HTML]{000000} \color{black} 1.03 & {\cellcolor[HTML]{F7FBFF}}
    \color[HTML]{000000} \color{black} 0.83 & {\cellcolor[HTML]{F1F7FD}}
    \color[HTML]{000000} \color{black} 0.93 & \color{black} 1.27 \\
    Spain (ES) & \color{black} 49 & {\cellcolor[HTML]{EAF2FB}} \color[HTML]{000000}
    \color{black} 0.91 & {\cellcolor[HTML]{F7FBFF}} \color[HTML]{000000}
    \color{black} 0.75 & {\cellcolor[HTML]{BFD8ED}} \color[HTML]{000000}
    \color{black} 1.40 & {\cellcolor[HTML]{6AAED6}} \color[HTML]{F1F1F1}
    \color{black} 1.94 & \color{black} 1.25 \\
    India (IN) & \color{black} 226 & {\cellcolor[HTML]{C8DCF0}}
    \color[HTML]{000000} \color{black} 1.27 & {\cellcolor[HTML]{6AAED6}}
    \color[HTML]{F1F1F1} \color{black} 1.52 & {\cellcolor[HTML]{F7FBFF}}
    \color[HTML]{000000} \color{black} 1.04 & {\cellcolor[HTML]{DFEBF7}}
    \color[HTML]{000000} \color{black} 1.16 & \color{black} 1.25 \\
    Belgium (BE) & \color{black} 31 & {\cellcolor[HTML]{6CAED6}}
    \color[HTML]{F1F1F1} \color{black} 1.56 & {\cellcolor[HTML]{C6DBEF}}
    \color[HTML]{000000} \color{black} 1.26 & {\cellcolor[HTML]{F7FBFF}}
    \color[HTML]{000000} \color{black} 0.95 & {\cellcolor[HTML]{F7FBFF}}
    \color[HTML]{000000} \color{black} 0.95 & \color{black} 1.18 \\
    Turkey (TR) & \color{black} 114 & {\cellcolor[HTML]{6CAED6}}
    \color[HTML]{F1F1F1} \color{black} 1.29 & {\cellcolor[HTML]{F7FBFF}}
    \color[HTML]{000000} \color{black} 0.96 & {\cellcolor[HTML]{7AB6D9}}
    \color[HTML]{000000} \color{black} 1.27 & {\cellcolor[HTML]{E3EEF9}}
    \color[HTML]{000000} \color{black} 1.02 & \color{black} 1.14 \\
    Viet Nam (VN) & \color{black} 252 & {\cellcolor[HTML]{6AAED6}}
    \color[HTML]{F1F1F1} \color{black} 1.53 & {\cellcolor[HTML]{DFEBF7}}
    \color[HTML]{000000} \color{black} 0.89 & {\cellcolor[HTML]{87BDDC}}
    \color[HTML]{000000} \color{black} 1.42 & {\cellcolor[HTML]{F7FBFF}}
    \color[HTML]{000000} \color{black} 0.67 & \color{black} 1.13 \\

    \midrule
    {\bf Global} & \color{black} 7,428 & {\cellcolor[HTML]{6AAED6}}
    \color[HTML]{F1F1F1} \color{black} 3.10 & {\cellcolor[HTML]{DCE9F6}}
    \color[HTML]{000000} \color{black} 1.83 & {\cellcolor[HTML]{F7FBFF}}
    \color[HTML]{000000} \color{black} 1.77 & {\cellcolor[HTML]{F7FBFF}}
    \color[HTML]{000000} \color{black} 1.60 & \color{black} 2.07 \\
    
  \bottomrule
  \end{tabular}
  }
  \caption{
  Top-25 countries with over 25 resolver pairs and the highest average
  rates of DNS censorship across both query types and network interfaces. Rates
  are expressed as the percentage of censored DNS queries.
  %
  Darker shaded cells indicate a higher rate of DNS censorship (compared to the
  country's average).  %
  The average rate of censorship for a country is computed across all four
  IP/query combinations.
  %
  The global row contains the mean of each column and includes data from the
  countries with less than 25 resolvers. These means weigh the contribution of
  each country equally, rather than weighted by the number of resolvers used in
  tests.   }
  \label{tab:prevalence:rates}
\end{table}



\CountriesCorporate

\para{How common is DNS censorship?} 
Our data shows the mean base rate of DNS censorship among the 106 countries
included in our study, across all query and network types, for our list of
domains is 2.1\%. 
%
A summary of the base rates observed in each of our four {\tt A/AAAA}-IPv4/IPv6
combinations for the 25 countries which have at least 25 pairs of resolvers that
were tested and perform the most censorship is illustrated in
\Cref{tab:prevalence:rates}. 
%
In general, our results concur with prior work which has also found high levels
of DNS censorship in China, Iran, Russia, and Hong Kong.

\para{Trends in censorship of IPv6-related queries in heavily censoring
countries.}
By measuring IPv6-related behaviors of censorship mechanisms, we uncover
a large number of DNS censorship inconsistencies in heavily censoring
countries.
%
Because our dataset is balanced (\ie all the domains have both {\tt A} and
{\tt AAAA} records and all the tested resolvers have an IPv4 and IPv6
interface), if censorship is independent of the query type and interface, we
expect to see a uniform rate of censorship across all query types and interface
combinations in \Cref{tab:prevalence:rates}. 
%
However, we find that this is not true. We observe two trends common to many of
the countries performing the most censorship. 
%
First, we see that, in comparison to any other query-interface combination,
{\em {\tt A} queries sent over IPv4 are the most heavily censored}. Second, we
see that {\em{\tt AAAA} queries are censored less than {\tt A} queries,
regardless of whether they are sent over IPv4 and IPv6 networks}.
%
This immediately suggests that censorship apparatus in a large number of
censoring regions are not fully IPv6-capable. Besides the possibility of
misconfiguration of the DNS censorship mechanisms, this may also be because
censors in these regions are not yet widely deployed on IPv6 networks in their
country.
%
The only exception to both these generalizations is China --- the most
censoring country in our data. China shows an unusual {\em preference towards
blocking {\tt AAAA} records regardless of whether they are sent over IPv4 or
IPv6}. We explore China and several other countries with
interesting patterns as specific case studies in \Cref{sec:cases}.

%% Moving all this to the case studies in cases.tex
%    \paragraph{Thailand} One prominent example is Thailand, where 8.2\% of
%    \texttt{A} records were blocked by IPv4 resolvers, compared to around 1\% for
%    \texttt{AAAA} records or IPv6 resolvers in the country. The reason for this is
%    that many instances of IPv4 resolvers predominantly (or exclusively) saw
%    blocking on only \texttt{A} records, while their IPv6 pair saw little or no
%    blocking on either \texttt{A} or \texttt{AAAA} records. For instance, one of the
%    larger resolvers by blocked domains had an IPv4 endpoint that blocked
%    240 \texttt{A} records, but only 15 \texttt{AAAA} records from our domain list.
%    Meanwhile, its IPv6 counterpart did not block any domains (either \texttt{A} or
%    \texttt{AAAA} records).
%    
%    % jq -r 'select(.resolver_country=="TH" and .id=="4934-A").blocked_domains | keys | .[]' ../../data/resolver-blocks-jan-aaaa.json | grep -e '-AAAA$' | wc -l
%    % Removed 3 domains manually: www.stratcom.mil, www.hud.gov, portal.hud.gov
%    
%    
%    
%    
%    \paragraph{China}
%    China shows an unusual preference \textbf{toward} blocking \texttt{AAAA}
%    records over \texttt{A} records. While it is known that the Great Firewall
%    can block \texttt{AAAA} records and injects IPv6 traffic, it is not clear why it
%    would block those records more than \texttt{A} records. Manual investigation reveals
%    21 domains that almost exclusively have their \texttt{AAAA} record blocked, but
%    not their \texttt{A} record. For example, \texttt{gmail.com}'s \texttt{AAAA}
%    record is blocked by over 95\% of resolvers in China, but the corresponding
%    \texttt{A} record for \texttt{gmail.com} is only blocked by 1\% of resolvers.
%    The other 20 domains have similar patterns, leading to China's slight preference
%    in blocking \texttt{AAAA} records over \texttt{A} records. We do not find any
%    instances of domains in China that are similarly exclusively blocked by
%    \texttt{A} record but not \texttt{AAAA}.
%    
%    %{"domain":"www.gmail.com-AAAA","country_code":"CN","v4_censored_count":186,"v6_censored_count":185}
%    %{"domain":"www.gmail.com-A","country_code":"CN","v4_censored_count":4,"v6_censored_count":1}
%    
%    %{"domain":"www.apple.com-AAAA","country_code":"CN","v4_censored_count":189,"v6_censored_count":191}
%    %{"domain":"www.apple.com-A","country_code":"CN","v4_censored_count":5,"v6_censored_count":8}
%    
%    % ~/v6/v4vsv6/cmd/baseRate$ python3 q6.py < ./CN.json 
%    %blocked in AAAA but not A: 21 domains
%    %translate.google.com
%    %learndigital.withgoogle.com
%    %www.snapchat.com
%    %news.google.com
%    %play.google.com
%    %video.google.com
%    %picasa.google.com
%    %www.projectbaseline.com
%    %googleblog.com
%    %hangouts.google.com
%    %messages.android.com
%    %www.apple.com
%    %android.com
%    %google.dz
%    %raseef22
%    %qalam.withgoogle.com
%    %www.gmail.com
%    %doubleclick.net
%    %gmail.com
%    %www.hacktivismo.com
%    %allo.google.com
%    %blocked in A but not AAAA: 0 domains
%    
%    
%    
%    
%    
%    \paragraph{Iran}
%    While Iran supports both IPv6 and IPv4, it is more effective at blocking IPv4
%    resolvers (both \texttt{A} and \texttt{AAAA} records). We find this is due to
%    many of the IPv6 resolvers in Iran are actually \textbf{6to4 bridges}: these
%    are IPv6 addresses that are not native IPv6, but instead an encoding of an IPv4
%    address. For example, sending an IPv6 packet to \texttt{2002:0102:0304::} will
%    send the packet encapsulated in an IPv4 packet to 1.2.3.4 (at the 6to4 gateway),
%    whose 32-bit address is encoded in the IPv6 address (\texttt{0102:0304}).
%    However, this can result in the packet passing encapsulated past the censor, who
%    will observe an IPv4 packet with a protocol field denoting IPv6 encapsulation,
%    instead of the usual UDP-carrying DNS. A naive censor may ignore such packets,
%    allowing the domain lookup through unchecked uncensored. We observe 272 IPv6
%    resolvers are actually 6to4 bridges in Iran, all of which block at lower
%    (but non-zero) rates, compared to their corresponding IPv4 resolver.
%    %The reason
%    %that these hosts still see blocking at all is that they occasionally must
%    %recursively resolve the host, which may induce censorship on the outbound
%    %packet. % (how did they get a correct cached entry in the first place tho?)
%    
%    % $ cat ../../data/resolver-blocks-jan-aaaa.json | jq '[.resovler_ip, .resolver_country] | @tsv' -r | awk '{print $1, $2}' | grep '2002:' | awk '{print $2}' | sort | uniq -c | sort -rn | head
%    %    628 KR
%    %    467 US
%    %    364 FR
%    %    272 IR
%    %    255 RU
%    %    244 DE
%    %    225 VN
%    %    175 IN
%    %    146 CA
%    %    127 GB
%    %
%    
%    
%    
%    % 5.56.132.126 2002:538:847e::538:847e
%    
%    \paragraph{Trends}
%    We observe two general trends that apply to many censoring countries in our data.
%    First, \textbf{IPv4 resolvers are more heavily censored than IPv6 resolvers}.
%    This may be because censors in those regions only inspect IPv4 packets, or that
%    censors are more widely deployed on IPv4 networks. Second, \textbf{Native record
%    types are more heavily censored than non-native records}. In other words, an
%    \texttt{A} record requested from an IPv4 resolver is more likely to be censored
%    than a \texttt{AAAA} record from the same interface. But conversely, we find
%    that \texttt{AAAA} records are also more often censored on IPv6 resolvers than
%    \texttt{A} records. This could be because a censor only supports detecting
%    \texttt{AAAA} records in IPv6 traffic, and only \texttt{A} records in IPv4.
%    
%    
%    \if0
%    Notably, China, Russia, Iran, and Hong Kong can clearly be seen to host
%    resolvers that consistently block many domains, demonstrating these countries have
%    deployed largely uniform national censorship policies. On the other hand,
%    countries that are known to censor may still appear to have no uniformly blocked
%    domains. This could be due to a country primarily using a different technology
%    to censor (such as HTTP, SNI, or IP blocking), bias in the set of domains we
%    queried, or in the set of resolvers used.
%    
%    One illustrative example is India, a country where widespread Internet censorship has
%    been observed and studied~\cite{singh2020india,Yadav2018a}. Given the number
%    of resolvers we discover and the country's proclivity to censor, we would expect to see
%    some domains censored across the country, but instead our data suggests no
%    domains are censored uniformly. One explanation for this discrepancy is that
%    India's censorship is not as centralized as in other countries: private ISPs in
%    India are given lists of URLs to block, but it is up to the ISP how to carry out
%    this blocking~\cite{Gosain2017a}, leading to heterogeneity in what is blocked,
%    what techniques are used, and how often updates occur.  Figure~\ref{fig:india}
%    shows a clustering of Indian resolvers by what domains they block, illustrating the non-uniformity in
%    conducting censorship in the country. We draw resolvers as nodes sized proportional
%    to their blocklist size, and draw an edge between two resolver
%    nodes if the set of domains they block has a high similarity (Levenshtein edit
%    distance less than a threshold). The largest censored list found in India contained 31
%    censored domains, while many other resolvers censored few or no domains.
%    % jq 'select(.resolver_country=="IN" and .id=="4335-A").blocked_domains | keys | .[]' ../../data/resolver-blocks-jan-aaaa.json -r | awk -F'-' '{print $1}' | sort | uniq | wc -l
%    
%    
%    \FigIndiaCluster
%    \fi
%    
%    

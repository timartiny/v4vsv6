\section{Discussion and Conclusions} \label{sec:discussion}

The design and evaluation of this experiment included many complicating factors.
Our decision making was structured to achieve the conservative result at each
stage. However, we provide a discussion of the limitations of our
experiments in order to transparently contextualize our work.

This context provides an interesting vantage into future directions for
pursuing censorship circumvention tools that leverage the gap in the transition
from IPv4 to IPv6. The inclusion of resolvers relying on 6to4 bridges demonstrates
that the challenge facing censors in thoroughly covering the transition from IPv4
to IPv6 is non-trivial. In this section we evaluate the limitations of our
measurements and analyses before proposing
related future research directions and concluding.

\subsection{Limitations}
\label{sec:discussion:limitations}

\para{External sources of data}
We draw much of the initial state for conducting our experiments and analyzing
the results from previous work. This includes the target domain selection as
outlined in \Cref{sec:methodology:domains} which relies on the accuracy of
the Satellite project. Similarly we use supplemental sources of information to
classify and categorize our data. We specifically rely on the accuracy of the
MaxMind geolocation and network classification datasets to inform our
understanding of each open resolver. We alse rely on the McAfee URL ticketing
system to provide categorical information about our selected target domains.
Our results and supported analyses are limited by the reliability of these
sources of information.

\para{Statistical Testing}
We filter many countries out of our statistical analysis due to the low sample
size of available resolvers within the country. We choose specific
bounds at which to perform analysis using the Z-test given that we cannot guarantee
that the resolver and domain entropy evaluation draws from a normal distribution.
With respect to available resolvers within a country we draw the lower bound at
25 available resolvers that pass our control tests. We believe that this provides
our statistical tests with appropriate power to draw significant conclusions.

\para{Resolver Selection}
While we followed the initial steps of Hendricks et al. in our resolver discovery
methodology (outlined in \Cref{sec:methodology:resolvers}), we did not perform
the follow up scanning to determine that each of the resolver pairs were truly
dual-stack. This leaves in open resolvers that forward or distribute queries as
long as long as the control queries are satisfied and the pairs geolocate to the
same country.

On one hand this can provide an inflated count of open resolver pairs within a
given country. On the other hand, real clients relying on open resolvers will
exercise this infrastructure as well -- as such, it is an important part of the DNS
ecosystem to understand.

One specific category of open resolver that we find to be relatively common
in our collected data are those that rely on a \textbf{6to4 Bridge}. This
configuration is identifiable as the resolvers IPv6 address will be drawn from 
the \texttt{2002::/16} address
block and has a discernable address format~\cite{RFC3056}.
The 6to4 protocol was originally designed as a mechanism to bridge the IPv6
deployment gap as public 6to4 bridges would provide interoperability
to legacy systems such that IPv4 services could communicate with IPv6
networks and vice-versa. The protocol works by assigning a \texttt{\textbackslash 48} subnet
with the encoded local address to an interface then allowing that
interface to receive traffic encapsulated in an IPv4 header at the local address.
In this way clients without outgoing connection to an IPv6 capable network
can send IPv6 packets over to an IPv6 capable host who forwards them onward.

We find multiple countries with limited IPv6 deployment rely on 6to4 resolvers.
Two specific examples are Iran in which 272 of the 277 resolvers identified used
6to4 and Thailand with 112 out of 186 resolvers. This
understanding of the protocols that are traversing the censored links provide
a limitation on our measurement in the sense that we do not get the comparison of
IPv4 with IPv6 that we hoped. However, they do provide an interesting perspective
on the ability of censors to handle protocols specifically deployed to assist in
bridging the IPv4 - IPv6 gap. For example in Thailand the resolvers relying on
6to4 censor requests at lower rates across the board.

% Do the non-6to4 resolvers in bangladesh and iran cluster by domain with the
% IPv4 censoring resolvers?

\subsection{Future Work}
\label{sec:discussion:future_work}

Given the identified discrepancy in censorship effectiveness between native IPv6
addresses and 6to4 bridges a promising direction for future
work is an investigation of IP tunneling protocol usage for censorship circumvention.
Our data and particularly Table ~\ref{tab6to4} in the Appendix show that 6to4 is
censored at rates different from national averages in countries known to censor.
Further, 6to4 is not the only protocol that has seen deployment as part of an
effort to bridge the gap between IPv4 and IPv6 deployment. The \texttt{6rd} protocol
extends the subnet range of 6to4. The Teredo
protocol uses UDP to encapsulate IP traffic. Finally the ISATAP protocol provides
an extended 6to4 adding configuration using multicast and DNS.
This strategy is not only applicable to DNS censorship circumvention as tunneling
protocols can be used to encapsulate any next layer.

%% It is not the only protocol designed to bridge the gap
% 6to4 - relatively static unicast encapsulation in extra IP header
% 6rd - extends subnets for 6to4
% teredo - Encapsulate in UDP
% ISATAP - provides 6to4+ using IPv6 packets encapsulated in IPv4 header and configuration using multicast and DNS.

We also propose directly extending the measurement that we have made to be
included in global longitudinal measurement studies examining DNS. The methodology
that we establish supports sustainable evaluation of global DNS censorship.

\subsection{Conclusions}

The current state of global IPv6 DNS censorship does not divide into a binary
block or allow, the landscape is nuanced and developing. However, we identify
clear trends in IPv6 censorship that can help motivate future research and
circumvention tool development.

In this work we present the first large scale measurement of DNS injection based
censorship in IPv6 and provide a reproducible methodology for incorporating this
measurement into longitudinal measurements going forward. We find significant
evidence of DNS injection in IPv6 in many countries around the world.
We find that countries favor traditional network infrastructure, censoring
\texttt{A} queries over IPv4 at the highest rates. At the country level we
use similarities and discrepancies in the censorship behavior across resolvers
and domains to identify censorship strategies. This allows us to demonstrate
various countries where censorship is likely centrally coordinated, as compared
with countries that demonstrate multiple clusters indicating looser coordination
between the entities enacting the DNS censorship regimes.

The IP layer and the gap that exists between full IPv6 deployment and the
current state of the Internet provide a telling lens for evaluating censorship
and developing censorship circumvention technologies.
IPv6 plays a growing role in global connectivity and, wherever possible,
censorship measurements should endeavor to include IPv6 analysis to provide the
most actionable understanding of global censorship infrastructure.

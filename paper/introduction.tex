\section{Introduction}\label{sec:intro}

%% hook
\para{The IPv6 transition introduces policy inconsistencies.}
The deployment of the IPv6 standard is now mainstream with nearly 35\% of
current Internet traffic being served over native IPv6 connections
\cite{Google-IPv6}. However, regardless of the recent increases in IPv6
availability and deployments, it is expected that a complete transition to IPv6
is unlikely to occur in the near future \cite{Prince-CF2021, Huston-APNIC2021}.
%
Consequently, this slow transition from the IPv4 to IPv6 protocol has
encumbered network operators with the task of supporting, securing, and
simultaneously maintaining both types of networks which have fundamentally
different performance characteristics \cite{perf}, routing behaviors and
network topologies \cite{topo}.
%
% highlight prior work
In fact, prior research has shown that such maintenance of dual-stack
deployments can be challenging and result in inconsistent network policy
enforcement.
%
For example, Czyz \etal \cite{Czyz-NDSS2016} found that dual-stack end-hosts
were found to have unintentional and significant access control policy
inconsistencies when comparing their IPv4 and IPv6 interfaces. 

%% Our angle
\para{Policy inconsistencies have the potential to benefit censorship
circumvention tools.}
Despite the concerns prompted by the identification of such policy
inconsistencies, we recognize that their presence may also present unique
opportunities to developers of censorship circumvention tools. For example,
capabilities of tools may be improved by opportunistically leveraging the IPv6
protocol when a censorship directive is implemented with IPv4 infrastructure
but not IPv6 infrastructure.
%
% why do we care
Making such gains, even in the short-term, is important given the arms race
nature of censorship and its circumvention \cite{reactive}.
%
%% what gap exists
What is lacking, however, is a data-driven understanding of what
inconsistencies exist in information controls deployments and, more generally,
how network operators tasked with maintaining information controls and
censorship deployments are handling the v4-to-v6 transition. 

\para{Our contributions.}
% one-liner of our work
Our work reduces this gap in knowledge by identifying inconsistencies in DNS
censorship over IPv4 and IPv6 infrastructure. In doing so, we make the
following contributions:

\begin{itemize}

  \item {\it Methodological improvements (\Cref{sec:methods}).} 
    We begin by building on the DNS censorship identification approaches used
    by prior work. More specifically, we take steps to improve the reliability
    of DNS censorship identification by integrating data from the TLS protocol
    and filtering unreliable DNS resolvers.

  \item {\it Measure prevalence of DNS censorship (\Cref{sec:prevalence}).} 
    We conduct a large-scale measurement of DNS censorship in six countries
    with IPv6 connectivity. We find \rnnote{fill in nice result summary here.}

  \item {\it Characterize differences due to DNS query type
    (\Cref{sec:query}).} 
    We then focus on the differences observed in DNS censorship based on
    whether the DNS query requested an IPv4 ({\tt A} record) or IPv6 address
    ({\tt AAAA} record). We find that \rnnote{summary}.

  \item {\it Characterize differences due to network infrastructure
    (\Cref{sec:dual}).}
    Finally, we focus on DNS censorship policy differences that occur in
    dual-stack resolver deployments --- \ie where the censorship determination
    is different for a resolver's IPv4 and IPv6 interfaces. We find that
    \rnnote{summary}.

\end{itemize}

Taken all together, this study provides a first look at IPv6 DNS censorship and
the policy gaps that arise from the IPv6 transition. We expand on the
implications of our research, for circumvention tool developers and measurement
researchers, in \Cref{sec:discussion}.

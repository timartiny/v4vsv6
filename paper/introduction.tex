\section{Introduction}\label{sec:intro}

Internet censorship is a global problem that affects over half the world's
population. Censors rely on sophisticated network middleboxes to inspect and
block traffic. A core component of Internet censorship is DNS blocking, and
prior work has extensively studied how DNS censorship occurs, both for specific
countries~\cite{Anonymous2020:TripletCensors,USESEC21:GFWatch} and
globally~\cite{kuhrer2015going,dagon2008corrupted,pearce2017global,scott2016satellite}.
%
%\cite{kuhrer2015going,Anonymous2020:TripletCensors,USESEC21:GFWatch,dagon2008corrupted,pearce2017global,scott2016satellite}.
These studies generally perform active measurements of DNS resolvers for large
sets of domains, and identify forged censorship responses from legitimate ones.

Unfortunately, prior work has focused exclusively on the IPv4 Internet, in part
because scanning the IPv6 Internet for open resolvers is
difficult~\cite{murdock2017target}, owing to its impossible-to-enumerate 128-bit
address space.
%
In this paper, we perform the first comprehensive global measurement of DNS
censorship on the IPv6 Internet. We leverage a recent network measurement
technique that can discover dual-stack IPv6 open resolvers from their IPv4
counterpart~\cite{hendriks2017potential}, and use these IPv4-IPv6 resolver pairs
to study DNS censorship globally. By sending DNS queries to both the IPv4 and IPv6
interfaces of the \emph{same resolver}, we measure the difference in
censorship on the IPv4 and IPv6 Internet.

\medskip

% \paragraph{Why study IPv6 censorship?}
IPv6 is becoming more widely deployed, with nearly 35\% of current Internet
traffic being served over native IPv6 connections~\cite{Google-IPv6}. However,
IPv6 has fundamentally different performance
characteristics~\cite{Dhamdhere-IMC2012}, network security
policies~\cite{Czyz-NDSS2016}, and network topologies~\cite{Czyz-SIGCOMM2014}
compared to the traditional IPv4 Internet.

While it may seem that censors either do or don't support detecting and
censoring IPv6 DNS in an all-or-nothing fashion, we find that there is a
tremendous range of how well a censor blocks in IPv6 compared to IPv4.
%
In particular, although nearly all of the countries we study have some support
for IPv6 censorship, we find that most block less effectively in IPv6 compared
to IPv4. For instance, we observe Thailand censors on average 80\% fewer IPv6 DNS
resources compared to IPv4 ones, despite a robust nation-wide censorship
system~\cite{gebhart2017internet}.


Studying censorship in IPv6 can provide opportunities for circumvention tools.
By identifying ways that censors miss or incorrectly implement blocking, we can
offer these as techniques that tools can exploit. Moreover, because of the
complex and heterogeneous censorship systems censors operate, many of these
techniques would be costly for censors to prevent, requiring investing
significant resources to close the IPv4/IPv6 gap in their networks. For this
reason, we believe IPv6 can provide unique techniques for circumvention
researchers and tool developers alike, that will be beneficial in the short term and
potentially robust in the longer term.

\medskip
% \paragraph{Findings}
We find a significant global presence of IPv6 DNS censorship --- comparable, but not
identical to well documented IPv4 censorship efforts. Censors demonstrate a
clear bias towards IPv4, censoring \texttt{A} queries in IPv4 at the highest rates,
and a propensity for censoring native record types (\texttt{A} in IPv4, \texttt{AAAA} in IPv6).
At the country level we break down differences by resolver and
domain across resource record and interface type. We find that multiple countries
--- Thailand, Myanmar, Bangladesh, Pakistan, and Iran ---
present consistent discrepancies across all resolvers or domains indicating
centrally coordinated censorship, where the policies that govern IPv4 and IPv6
censorship are managed centrally.
Other countries show more varied discrepancies in the
ways that resolvers censor IPv4 and IPv6, due to decentralized models of
censorship, such as that in Russia~\cite{ramesh2020decentralized}, or due to
independent and varied corporate network firewalls.

We also identify behavior indicative of censorship oversight that can be advantageous
to censorship circumvention. For example Iran censors queries that rely on 6to4
bridges at significantly lower rates, presumably due to the encapsulation of an
IPv6 DNS request in an IPv4 packet, instead of appearing as UDP.


Taken all together, this study provides a first look at IPv6 DNS censorship and
the policy gaps that arise from the IPv6 transition. We provide the following
contributions:

\begin{itemize}
    \item
    %We perform the first comprehensive study on IPv6 DNS censorship
    We conduct the first large-scale measurement of IPv6 DNS censorship in
    over 100~IPv6-connected countries. %We find \rnnote{fill in nice result summary here.}
    We find that while most censors support IPv6 in some capacity, there are
    significant gaps in how well they censor IPv6.


    \item We provide methodological improvements on measuring DNS censorship
    that avoids relying on cumbersome IP comparisons (that are not
    robust to region-specific DNS nameservers). Our methods are easily
    reproducible, and can be used in future measurement studies

    \item We characterize the difference in censorship of both network type
    (IPv4 and IPv6), and resource type ({\tt A} and {\tt AAAA} record), and
    identify trends in several countries.

    \item Using our findings, we suggest several new avenues of future
    exploration for censorship circumvention researchers, and censorship
    measurements.

\end{itemize}


The remainder of this paper is organized as follows. \Cref{sec:background} provides
background information in DNS censorship, and the relation of IPv6 to relevant DNS
infrastructure. We outline our compiled methodology and ethical design
considerations in \Cref{sec:methodology} before presenting our findings on the
global prevalence of IPv6 censorship in \Cref{sec:prevalence}. We then dig into
per country analysis based on Resource Record types in \Cref{sec:resources} and
IP protocol version in \Cref{sec:infrastructure}. We select several case studies
to highlight in \Cref{sec:cases} before covering related work in
\Cref{sec:related_work}. Finally \Cref{sec:discussion} provides discussion and
contextualization of this work before concluding.

% Censorship is bad
% Censorship measurements don't hurt
% We study censorship in the context of IPv6. Why?
% -Not all or nothing: censors can deploy v6 blocking, but be worse at it
% -Not short term: censors may have to invest significant resources to be better
% -Given above, circumvention tools can benefit
% High level takeaways: some censors bad at IPv6
% -Clear bias toward IPv4, though most censors have some IPv6 capabilities
% -

\if0
%% hook
\para{The IPv6 transition introduces policy inconsistencies.}
The deployment of the IPv6 standard is now mainstream with nearly 35\% of
current Internet traffic being served over native IPv6 connections
\cite{Google-IPv6}. However, regardless of the recent increases in IPv6
availability and deployments, it is expected that a complete transition to IPv6
is unlikely to occur in the near future \cite{Prince-CF2013, Huston-APNIC2021}.
%
Consequently, this slow transition from the IPv4 to IPv6 protocol has
encumbered network operators with the task of supporting, securing, and
simultaneously maintaining both types of networks which have fundamentally
different performance characteristics \cite{Dhamdhere-IMC2012}, routing
behaviors and network topologies \cite{Czyz-SIGCOMM2014}.
%
% highlight prior work
In fact, prior research has shown that such maintenance of dual-stack
deployments can be challenging and result in inconsistent network policy
enforcement.
%
For example, Czyz \etal \cite{Czyz-NDSS2016} found that dual-stack end-hosts
were found to have unintentional and significant access control policy
inconsistencies when comparing their IPv4 and IPv6 interfaces. 

%% Our angle
\para{Policy inconsistencies have the potential to benefit censorship
circumvention tools.}
Despite the concerns prompted by the identification of such policy
inconsistencies, we recognize that their presence may also present unique
opportunities to developers of censorship circumvention tools. For example,
capabilities of tools may be improved by opportunistically leveraging the IPv6
protocol when a censorship directive is implemented with IPv4 infrastructure
but not IPv6 infrastructure.
%
% why do we care
Making such gains, even in the short-term, is important given the arms race
nature of censorship and its circumvention \cite{Tschantz-SP2016}.
%
%% what gap exists
What is lacking, however, is a data-driven understanding of what
inconsistencies exist in information controls deployments and, more generally,
how network operators tasked with maintaining information controls and
censorship deployments are handling the v4-to-v6 transition. 

\para{Our contributions.}
% one-liner of our work
Our work reduces this gap in knowledge by identifying inconsistencies in DNS
censorship over IPv4 and IPv6 infrastructure. More specifically, we make the
following contributions.
% In doing so, we make the following contributions:

\begin{itemize}

%   \item {\it Methodological improvements (\Cref{sec:methods}).} 
%     We begin by building on the DNS censorship identification approaches used
%     by prior work. More specifically, we take steps to improve the reliability
%     of DNS censorship identification by integrating data from the TLS protocol
%     and filtering unreliable DNS resolvers.

  \item {\it Measure prevalence of DNS censorship (\Cref{sec:prevalence}).} 
    We conduct the first large-scale measurement of IPv6 DNS censorship in
    \fixme{XX} IPv6-connected countries. We find \rnnote{fill in nice result summary here.}

  \item {\it Characterize differences due to resource record types
    (\Cref{sec:resources}).}
    We then focus on the differences observed in DNS censorship based on
    whether the DNS query requested an IPv4 ({\tt A} record) or IPv6 address
    ({\tt AAAA} record). We find that \rnnote{summary}.

  \item {\it Characterize differences due to network infrastructure
    (\Cref{sec:infrastructure}).}
    Finally, we focus on DNS censorship policy differences that occur in
    dual-stack resolver deployments --- \ie where the censorship determination
    is different for a resolver's IPv4 and IPv6 interfaces. We find that
    \rnnote{summary}.

\end{itemize}

Taken all together, this study provides a first look at IPv6 DNS censorship and
the policy gaps that arise from the IPv6 transition. We expand on the
implications of our research, for circumvention tool developers and measurement
researchers, in \Cref{sec:discussion}.

\fi
